\chapter{Type equality constraints}

Our first extension will be to add type equality constraints to the CLI. 
Type equality constraints for C\# are described in \cite{gadts}, we will 
be using these ideas applied to the CLI not C\#. 

\section{Generalized algebraic data types}

The basic idea is to extend generic constraints to allow equality constraints 
on generic types. Section 3.1 (Equational constraints for C\#) of \cite{gadts} 
describes this extension. This extension would allow a generic type to be declared equal 
to another type, and be statically checked at compile time. For example,
a list flatten method could check that the list was a list of lists by the 
addition of the \texttt{where T=List<U>} clause.

\begin{lstlisting}[caption={Type equality constraints in extended C\#},keywordstyle={\color{blue}},language=sharpc,tabsize=4]
public abstract class List<T> {
	...
	public abstract List<T> Append(List<T> list);
	public abstract List<U> Flatten<U>() where T=List<U>;
}

public class Nil<T> : List<T> {
	public override List<U> Flatten<U>() {
		return new Nil<U>;
	}
}

public class Cons<T> : List<T> {	
	T head; List<T> tail;
	public override List<U> Flatten<U>() {
		return this.head.Append(this.tail.Flatten());
	}
}
\end{lstlisting}

Calling \texttt{Flatten} on a \texttt{List<T>} would statically check
that \texttt{T=List<U>} where \texttt{U} is any type. Thus in the
method body of flatten we can assume that the type of \texttt{head}
is \texttt{List<U>} which has an \texttt{Append} method. While the
paper suggests this as a C\# extension generic constraints are currently
encoded at the CLI level and so we could add this as a CLI extension as well,
thus allowing this to be added to C\# and other languages easily. 

\subsection{Example}

The following shows a minimal list example using type equality constraints, in 
both extended C\# and CIL. We use the syntax from \cite{gadts} for the C\#
code, we use a similar syntax for the CLI code. The changes compared to
standard code are highlighted.

\begin{lstlisting}[caption={Type equality constraints in extended C\#\protect \\
Extension of listing \ref{lis:csharp gadt}},escapechar={~},keywordstyle={\color{blue}},language=sharpc]
public abstract class List<T>
{
	public abstract List<T> Append(List<T> list);
	public abstract List<U> Flatten<U>() ~\colorbox{yellow}{where T=List<U>;}~
}

public class Nil<T> : List<T>
{
	public override List<T> Append(List<T> list)
	{
		return list;
	}
	public abstract List<U> Flatten<U>() // type constraints are inherited
	{
		return new Nil<U>();
	}
}

public class Cons<T> : List<T>
{
	T Head;
	List<T> Tail;

	public Cons(T head, List<T> tail) 
	{
		Head = head;
		Tail = tail;
	}

	public override List<T> Append(List<T> list)
	{
		return new Cons<T>(Head, Tail.Append(list));
	}

	public override List<U> Flatten<U>() // type constraints are inherited
	{
	    return ~\colorbox{yellow}{Head.Append(Tail.Flatten<U>());}~ // invalid in standard C#
	}
}
\end{lstlisting}

This small example shows all the new features of type equality constraints. As
stated before, it is a small addition. The features are the addition of type
equality constraints after methods (\texttt{where T=List<U>}) and the ability
to treat a value of one type as another without having to cast
(\texttt{Head.Append(...)}).

\begin{lstlisting}[caption={Corresponding CIL},escapechar={~},language=cil]
.class public abstract auto ansi beforefieldinit List<T>
extends [mscorlib]System.Object 
{
	.method family hidebysig specialname rtspecialname instance void .ctor() 
		cil managed
	{
		.maxstack 8
		ldarg.0
		call instance void [mscorlib]System.Object::.ctor()
		ret
	}

	.method public hidebysig newslot abstract virtual instance class 
		Test.List`1<!T> Append(class Test.List`1<!T> list) cil managed     
	{     }

	.method public hidebysig newslot abstract virtual instance class 
		Test.List`1<!!U> Flatten<U>() cil managed ~\colorbox{yellow}{where T=List<U>}~
	{     }
}

.class public auto ansi beforefieldinit Nil<T>     
	extends Test.List`1<!T> 
{     
	.method public hidebysig specialname rtspecialname instance void .ctor() 
		cil managed     
	{
		.maxstack 8
		ldarg.0
		call instance void Test.List`1<!T>::.ctor()
		ret
	}

	.method public hidebysig virtual instance class 
		Test.List`1<!T> Append(class Test.List`1<!T> list) cil managed     
	{
		.maxstack 1
		ldarg.1
		ret
	}

	.method public hidebysig virtual instance class 
		Test.List`1<!!U> Flatten<U>() cil managed ~\colorbox{yellow}{where T=List<U>}~
	{
		.maxstack 1
		newobj instance void Test.Nil`1<!!U>::.ctor()
		ret
	}
}

.class public auto ansi beforefieldinit Cons<T>     
extends Test.List`1<!T> 
{
	.method public hidebysig specialname rtspecialname instance void 
		.ctor(!T head, class Test.List`1<!T> tail) cil managed     
	{         
		.maxstack 2        
		ldarg.0
		call instance void Test.List`1<!T>::.ctor()
		ldarg.0
		ldarg.1
		stfld !0 Test.Cons`1<!T>::Head
		ldarg.0
		ldarg.2
		stfld class Test.List`1<!0> Test.Cons`1<!T>::Tail
		ret
	}
	
	.method public hidebysig virtual instance class 
		Test.List`1<!T> Append(class Test.List`1<!T> list) cil managed     
	{         
		.maxstack 3
		ldarg.0
		ldfld !0 Test.Cons`1<!T>::Head
		ldarg.0
		ldfld class Test.List`1<!0> Test.Cons`1<!T>::Tail
		ldarg.1
		callvirt instance class Test.List`1<!0> 
			Test.List`1<!T>::Append(class Test.List`1<!0>)
		newobj instance void Test.Cons`1<!T>::.ctor(!0, class Test.List`1<!0>)
		ret
	}

    .method public hidebysig virtual instance class
		Test.List`1<!!U> Flatten<U>() cil managed ~\colorbox{yellow}{where T=List<U>}~ 
	{         
		.maxstack 2
		nop
		ldarg.0
		ldfld !0 Test.Cons`1<!T>::Head
		ldarg.0
		ldfld class Test.List`1<!0> Test.Cons`1<!T>::Tail
		callvirt instance class Test.List`1<!!0> 
			Test.List`1<!T>::Flatten<!!U>()

		// the following callvirt would not verify in the standard CLI
    ~\colorbox{yellow}{
		callvirt instance class Test.List`1<!0> 
			Test.List`1<!!U>::Append(class Test.List`1<!0>)}~
		ret
	}

    .field private !T Head
    .field private class Test.List`1<!T> Tail
}
\end{lstlisting}

The corresponding CLI code has very similar additions to C\#. Methods can have equality 
constraints, but note that they need to be redeclared, they are not automatically inherited. 
This is to keep a close mapping between the metadata and the syntax. Methods can be called 
on types that do not declare that method, but are equal to a type that does. This can be seen 
with the \texttt{callvirt} to \texttt{Append} instruction. At the point that instruction is 
called the stack slot that will be used for the \texttt{this} argument is of type \texttt{T}, a 
type that does not define \texttt{Append} and is not constrained to any interface or base class
that defines \texttt{Append}.

While there is no implementation of these extensions, the work done in
\cite{gadts} on formalization and its simple mapping to the CLI described here
should be enough to convince that this is a sensible, safe and correct extension 
of the type system.

\section{Changes to ECMA-335}

ECMA-335\cite{ecma-335} is the specification for the CLI. It defines everything
from the low level details of assembly file format to high level concepts such
as types, values and locations.

The specification is 548 pages long, and while not all of it is relevant to
equality constraints it is still a large system to understand. As equality
constraints share some similarities to generic constraints we begin with an
overview of them. The changes required however are fairly small and localized to 
a few areas of the specification. Assignment compatibility, method call verification 
and metadata tables are the only areas modified by the addition of type equality constraints.

\subsection{Generic constraints}

Section II.9.11 (Constraints on generic parameters)\cite{ecma-335} specifies
generic constraints. A type parameter that has been constrained must
be instantiated with an argument that is assignable to each of declared
constraints, and that satisfies all special constraints. 

The special constraints currently defined in the CLI are, \texttt{+}, \texttt{-}, \texttt{class},
\texttt{valuetype} and \texttt{.ctor}. \texttt{class} constrains the
argument to be a reference type. \texttt{valuetype} constrains the
argument to be a value type, except for any instance of\texttt{ System.Nullable<T>}.
\texttt{.ctor} constrains the argument to a type that has a public
default constructor (implicitly this means all value types as value
types always have a public default constructor). Finally \texttt{+}
and \texttt{-} are used to denote the parameter is covariant or contravariant
respectively.

While it might seem that this is a good place to add our extension
type equality constraints are a constraint on the entire parameter
list, not on each individual parameter as with generic constraints. 
For example a method \texttt{m<U,V>} could have a constraint relating
\texttt{U} and \texttt{V}, where it does not make sense to apply the 
constraint to either one of the parameters. There's also the potential
to add an equality constraint to a non-generic method.

\begin{lstlisting}[keywordstyle={\color{blue}},language=sharpc]
class Foo<T>
{
	public void Bar<U, V>(T list) where T = Pair<U, V>
	{
		...
	}
	public void Fizz(List<int> list) where T = int
	{
		...
	}
	
	public void Buzz(List<string> list) where T = string
	{
		...
	}
}
\end{lstlisting}

In this example \texttt{Bar} can only be called if \texttt{Foo<T>}
was initialized with \texttt{Pair<U,V>}, \texttt{Fizz} only if it
was initialized with \texttt{int}, and \texttt{Buzz} only if it was
initialized with \texttt{string}. A similar thing can be done with
non-generic inner types. 

So we need to look to add this new syntax somewhere separate to the 
generic parameter list. Preferably it would have similar syntax for 
both methods and types (as generic parameters look the same on a type 
declaration or method declaration). A type declaration currently follows 
the pattern:
\newline
``\texttt{.class ClassAttr{*} Id {[}'<' GenPars '>'{]} {[}extends TypeSpec {[} implements TypeSpec{]} {[}',' TypeSpec{]}{*}{]}}''
\newline
While method declarations follow the pattern:
\newline
``\texttt{.method MethAttr{*} {[}CallConv{]} Type {[}marshal '(' {[}NativeType{]} ')'{]} MethodName {[}'<' GenPars '>'{]} '(' Parameters ')' ImplAttr{*}}''.
\newline

Adding a new clause ``\texttt{where {[}Type '=' Type{[}',' Type '=' Type{]}{*}{]}}
to method declarations after the parameter list gives us a list of Types that must 
be equal to other types. It's not strictly necessary to have this clause on type 
declarations as for top level types it makes very little sense and for inner types 
it can be emulated be adding the clause to each method.

\subsection{Assignment compatibility}

Assignment compatibility is defined in section I.8.7 of \cite{ecma-335}, further to this 
verification assignment compatibility is defined in III.1.8.1.2.3. Verification assignment 
compatibility is mostly defined in terms of general assignment compatibility from I.8.7.3.

Verification assignment compatibility is used be the verifier for determining if method calls, 
field references and loads and stores are valid. This is decided based on the stack type and
signiture at each instruction. If verification assignment compatibility is extended to understand 
type equality constraints then operations that were unverifiable but type correct can now be 
checked as verifiable as well.

Adding another rule to \emph{verifier-assignable-to} to use the rules for equality constraints is all that is needed
to enhance this part of the system.

\textbullet{} T is \emph{equal-to} U.

\subsubsection{\emph{equal-to}}

\emph{equal-to} is used to determine if two type names refer to the same actual type.
It uses the both the global typing environment $\Gamma$ and the
equality constraints on the current method $\epsilon$ which defines equal types.

These rules are the same as the rules presented in Section 5 of
\cite{gadts}. 

$\Gamma$ is the typing environment, $\epsilon$ is the current set of equality
constraints.

The judgement $\Gamma,\epsilon \vdash T \, \mathit{ok}$ states that type $T$ is
well formed with respect to $\Gamma$.

The judgement $\Gamma,\epsilon \vdash T=U$ states that $T$ and $U$ are
equivalent with respect to the environment $\Gamma$ and the current equality
constraints $\epsilon$.

\begin{prooftree}
\AxiomC{$T=U \in \epsilon$}
\LeftLabel{eq-hyp}
\UnaryInfC{$\Gamma,\epsilon \vdash T=U$}
\end{prooftree}

\begin{prooftree}
\AxiomC{$T=U \in \epsilon$}
\AxiomC{$\Gamma \vdash C<T> \,\mathit{ok}$}
\LeftLabel{eq-con}
\BinaryInfC{$\Gamma,\epsilon \vdash C<T>=C<U>$}
\end{prooftree}

\begin{prooftree}
\AxiomC{$\Gamma,\epsilon \vdash C<T>=C<U>$}
\LeftLabel{eq-decon}
\UnaryInfC{$\Gamma,\epsilon \vdash T=U$}
\end{prooftree}

\begin{prooftree}
\AxiomC{$\Gamma \vdash T \,\mathit{ok}$}
\LeftLabel{eq-refl}
\UnaryInfC{$\Gamma,\epsilon \vdash T=T$}
\end{prooftree}

\begin{prooftree}
\AxiomC{$U=T \in \epsilon$}
\LeftLabel{eq-sym}
\UnaryInfC{$\Gamma,\epsilon \vdash T=U$}
\end{prooftree}

\begin{prooftree}
\AxiomC{$\Gamma,\epsilon \vdash T=U$}
\AxiomC{$\Gamma,\epsilon \vdash U=V$}
\LeftLabel{eq-sym}
\BinaryInfC{$\Gamma,\epsilon \vdash T=V$}
\end{prooftree}

\subsection{Method calls}

In addition to verification rules already present on method call instructions, 
method calls must now also be checked that any equality constraints on the method
being called can be satisfied by the current context. This is done by checking
that for each constraint listed on the method ($T=U$) T and U are
\emph{equal-to} in the current context after applying substitution for generic
parameters.

$\Gamma$ and $\epsilon$ have the same meaning as above. $m : <\bar{U}>$ means
that $m$ is a method with formal generic parameters $\bar{U}$. $m<\bar{T}>$ is
instantiation of method $m$ passing $\bar{T}$ as the actual generic parameters.

The judgement $\Gamma,\epsilon \vdash m<\bar{T}>$ states that the method call
to $m$ with actual generic parameters $\bar{T}$ is well formed in the context
given by $\Gamma$ and $\epsilon$.

\begin{prooftree}
\AxiomC{$\Gamma \vdash m : <\bar{U}> \,\mathit{where}\, \epsilon'$}
\AxiomC{$\forall e \in (\epsilon'[\bar{T}/\bar{U}]).\Gamma, \epsilon \vdash e$}
\LeftLabel{call}
\BinaryInfC{$\Gamma,\epsilon \vdash m<\bar{T}>$}
\end{prooftree}

This is an addition to the verifiability rules for the \texttt{call}, \texttt{calli} and \texttt{callvirt}
instructions. \texttt{call} and \texttt{calli} are defined in section III.3.19 and III.3.20, \texttt{callvirt}
is defined in section III.4.2. These sections describe the instructions operation as well as their correctness and 
verification rules.

\subsection{Metadata tables}

Metadata tables are specified in section 22 (Metadata logical format:
tables). We're mostly interested in section 22.21 (GenericParamConstraint),
which explains how generic constraints are stored in the assembly. 
\begin{quotation}
The GenericParamConstraint table has the following columns:

\textbullet{} Owner (an index into the GenericParam table, specifying
to which generic parameter this row refers) 

\textbullet{} Constraint (an index into the TypeDef, TypeRef, or TypeSpec
tables, specifying from which class this generic parameter is constrained
to derive; or which interface this generic parameter is constrained
to implement; more precisely, a TypeDefOrRef (�24.2.6) coded index) 

The GenericParamConstraint table records the constraints for each
generic parameter. Each generic parameter can be constrained to derive
from zero or one class. Each generic parameter can be constrained
to implement zero or more interfaces. 

Conceptually, each row in the GenericParamConstraint table is owned
by a row in the GenericParam table. 

All rows in the GenericParamConstraint table for a given Owner shall
refer to distinct constraints.
\end{quotation}
We need a similar table for our equality constraints. We can't however 
extend the GenericParamConstraint table as each row in that table is tied
to one method paramter, we need each row to be tied to one method. 
It will need an owner (a type or method) and then two type references 
(either a defined or referenced type, or a generic type parameter) that are constrained
to be equal within the owner. Another column that isn't necessary
but could be useful is a flags field describing the relationship between
the two types, currently this would always be set to equal, however
the system could be extended at a later date to allow less than and
greater than relationships as well.

So the TypeRelationshipConstraint table has the following columns:
\begin{itemize}
\item Owner (an index into the MethodDef table, specifying the
Method to which this constraint applies)
\item Flags (a 1-byte flag bitmask currently always set to 0. To be used
for extensions)
\item ConstraintA (an index into the TypeDef, TypeRef, or TypeSpec tables,
specifying the first type; more precisely, a TypeDefOrRef (�24.2.6)
coded index)
\item ConstraintB (an index into the TypeDef, TypeRef, or TypeSpec tables,
specifying the second type; more precisely, a TypeDefOrRef (�24.2.6)
coded index)
\end{itemize}

\section{Implementation changes}

This system could be implemented in the open source Mono runtime. The Mono
runtime is requires the majority of the changes to support this extension. The
reflection library also requires changing however it is mostly implemented via
internal calls to the runtime.

The runtime is written in C, and has around 410,000 lines of code (total lines
of C code is 414,181). Mono including all its supporting tools and libraries
comes to over 5,000,000 lines of code. While Mono is well organized trying to
learn a project of this size with no prior experience was optimistic. 

An implementation of this system was a goal of this project, but time
constraints, the complexity of Mono and our inexperience with the system has
resulted in us not having an implementation.

We have however spent a lot of time reading and attempting to understand the
Mono source code and feel that recording what needs to be changed at a high
level is useful information.

\subsection{Runtime}

The runtime needs to be changed to load and understand equality constraint
metadata. These changes will be focused in mono/metadata/metadata.h/c and
mono/metadata/reflection.h/c.  The TypeRelationshipConstraint table is very
similar to the GenericParamConstraint table, the current code for
GenericParamConstraint can guide the new code for TypeRelationshipConstraint.

Verification is done in verify.c. Additions need to be made to verify that type
relationship constraints are met when calling methods and to extend assignment
compatibility to use those type relationship constraints.  This requires
additions to \texttt{mono\_method\_verify} to check constraints on
\texttt{call}, \texttt{calli} and \texttt{callvirt} instructions, this can
probably be done in the procedure \texttt{do\_invoke\_method}.  It also requires additions to the procedure
\texttt{verify\_stack\_type\_compatibility\_full} to check the new assignment
compatibility rules.

\subsection{corlib}

The core library needs to be changed to support the metadata associated with type equality constraints. Specifically the
reflection library defined in mcs/class/corlib/System.Reflection\[.Emit\]. This
API is predominately implemented via internal calls that are then defined in
mono/metadata/icall-def.h.

\subsection{Cecil and IKVM}

Mono.Cecil and IKVM are well known libraries for working with CIL code. Used for
generating and analzing CIL code, they allow the reading, editing and writing of 
CIL assemblies. They are generally considered better than the core library System.Reflection and
System.Reflection.Emit. Any extensions of the system would be well served by
extending these libraries to understand the extension as well. This would allow 
tools that currently use Cecil and IKVM to target type equality constraints.

\subsection{Assembler}

While it would seem that changing the CIL assembler would be an integral
requirement of this extension it is not actually all that useful. Most CIL code
is generated via in-memory data structures using libraries such as Cecil and
then written out directly to binary format skipping the text stage, coupled
with the fact that hand written CIL is rare the assembler is not very
important. 

We also could not find up to date source code for a CIL assembler. The
assembler in the Mono source tree pre-dates .NET 2.0, not even supporting
generics!

\section{Test plan}

As part of any implementation, a number of test cases should be developed to
show that the system works. These tests need to show that there is no CLI code
that does not run in the new system (backwards compatibility) and that the new
system is correct.

Normally these tests would be written in CIL assembly, however writing tests in
CIL assembly would be time consuming and error prone. Instead, test writing
should be assisted by an assembly rewriter. The rewriter will look for calls 
to special methods and attributes in a CLI assembly and rewrite the bytecode and
metadata to use features in the extended system.  Assembly rewriting will be
done using the CLI metadata reflection, writing and rewriting via
\texttt{System.Reflection} or another metadata editor such as
\texttt{Mono.Cecil}. Any implementation will extend these objects to support
the new metadata constructs.  With this we can mark up standard C\# or F\# code
with special methods and attributes and rewrite the resulting assemblies to
include the new metadata.

This system of testing is not immeditely obvious. Discussions at the start of the project brought up the
issue that writing tests was neccessary but writing them in CIL was difficult. It was suggested early on
that we write tests in a high level language and modify the compiler to output the new CIL code. While this would 
work, the high level langauge compilers are another very complex system that we would rather not touch. Unreleated work
with calling native APIs from C\# using injected CIL led to the inspiration to do something similar for testing.

This system of testing has a number of advantages. Firstly it allows us to
write our tests in a high level language such as C\# instead of low level CIL.
The main benefit of writing in a high level language should be clear, it's much
easier. However we also get another less obvious benefit, that we are able to
test both imperative (C\#) and functional (F\#) code. This should mean coverage
over most aspects of the CLI.

Secondly we can run the tests on the standard CLI, this gives us something to
test against. Running in the standard CLI is not equivalent to running the
rewritten program in the new CLI (which we will call CLI+), but does allow some
deductions to be made.

For type equality there are two methods we need to define. \texttt{U Cast<T,
U>(T obj)} that checks \texttt{T} and \texttt{U} are equal at runtime and casts
\texttt{obj} to type \texttt{U} if they are; and \texttt{void EqualTypes<T,
U>()} that will check that T and U are equal types at runtime. Both these
methods will throw an exception \texttt{TypeEqualityException} if \texttt{T}
and \texttt{U} are not equal.

The rewriter will search an assembly for uses of \texttt{Cast<T, U>} and
\texttt{EqualTypes<T,U>}. Any use of \texttt{EqualTypes<T,U>} will be removed
and the method metadata rewritten to include the new type equality tags. Any
use of \texttt{Cast<T, U>} will also rewrite the metadata to include the new
type equality tags and will also remove the call to \texttt{Cast}.

\begin{lstlisting}[caption={Cast},keywordstyle={\color{blue}},language=sharpc]
public static class TypeEquality
{
	public static U Cast<T, U>(T obj)
	{
		if(typeof(T) == typeof(U))
		{
			return (U)(Object)obj;
		}
		else
		{
			throw new TypeEqualityException();
		}
	}
	public static void EqualTypes<T, U>()
	{
		if(typeof(T) != typeof(U))
		{
			throw new TypeEqualityException();
		}
	}
}
\end{lstlisting}

\subsection{Formal model}

A program $P$ is the source code in a CLI language, such as C\#. It may or may
not make use of \texttt{TypeEquality.Cast} and
\texttt{TypeEquality.EqualTypes}. It can be compiled by a compiler $C$ to give
a CIL assembly, this assembly can be run on a runtime to give a value. The two
runtimes are $CLI$ and $CLI+$, values are either some value $v$ representing
the overall act of computation done by the program or an exception thrown by
the program, the only exceptions we care about are $TypeEqualityException$, or
$VerificationException$.  Finally a CIL assembly can be rewritten by the
rewriter $R$ to produce a CIL+ assembly.

If $R$ and CLI+ are correct then the following statements should hold:
\begin{lemma}
$C(P) \underset{CLI}{\rightarrow} TypeEqualityException \implies R(C(P))
\underset{CLI+}{\rightarrow} VerificationException$
\end{lemma}
\begin{lemma}
$R(C(P)) \underset{CLI+}{\not\rightarrow} VerificationException \implies
C(P) \underset{CLI}{\not\rightarrow} TypeEqualityException$
\end{lemma}
\begin{lemma}
$C(P) \underset{CLI}{\rightarrow} v \wedge R(C(P))
\underset{CLI+}{\not\rightarrow} VerificationException \implies R(C(P))
\underset{CLI+}{\rightarrow} v$
\end{lemma}
\begin{lemma}
$C(P) \underset{CLI}{\rightarrow} v \iff C(P) \underset{CLI+}{\rightarrow} v$
\end{lemma}

Lemma 1 states that if a program throws a \texttt{TypeEqualityException} in
the standard CLI then it will throw a \texttt{VerificationException} in the new
CLI. If CLI+ doesn't throw a \texttt{VerificationException} then we know
something is wrong with either the rewriter or the new runtime. This property does not
hold in reverse \newline($R(C(P)) \underset{CLI+}{\rightarrow} VerificationException
\centernot \implies C(P) \underset{CLI}{\rightarrow} TypeEqualityException$) as the call
to \texttt{Cast<T, U>} or \texttt{EqualTypes<T, U>} might not be hit be every
control flow path.

Lemma 2 states that if the rewritten program runs in CLI+ and does not throw a
\texttt{VerificationException} then running the code in the CLI will not throw
a \texttt{TypeEqualityException}. If a \texttt{TypeEqualityException} is thrown
then something is wrong with the new runtime.

Lemma 3 states that if the program computes a value $v$ in the standard CLI
and verifies correctly in the new CLI then the new CLI should compute the same
value $v$.

Lemma 4 states that if the same CLI assembly (no rewriting) is run on both
runtimes they should compute the same value. While this is similar to lemma 3
it's difference is the lack of any assembly rewriting. 

All this together means that we can do some verification of our new system
against the old system. If a program ran correctly in the old system it should
also run correctly in the new system (lemma 4). If a program has correct equality type
constraints then it should run in the old system (and by the lemma 4 also
run in the new system) and when rewritten it should run in the new system
(lemma 3). If a program throws a \texttt{TypeEqualityException} then it should
throw a \texttt{VerificationException} when rewritten and run in the new
system.

\section{Testing without a full implementation}

While the above test plan is appropriate for checking an implementation is
correct, we do not have an implementation. However, we still wish to show that
the system could work and is statically checkable. To do this, we will be using
a similar system to above, and continue to write tests as described above.
However, instead of using an assembly rewriter to transform them into an
extended CIL format, we use an assembly analyzer to perform the checks that the extended verifier
would perform. Thus for every method call we check to see if it calls
\texttt{EqualTypes} and if so, check that we can satisfy those constraints at
all call sites, and for every call to \texttt{Cast} we check that the two types
are indeed equal, either directly or via equality constraints.
