\chapter{Type equality constraints}

\section{Generalized algebraic data types}

This report looks into how to add type equality constraints to the
CLI. Type equality constraints for C\# are described in \cite{gadts},
we'll be using these ideas but for the CLI not C\#. The basic idea
is to extend generic constraints to allow equality constraints on
generic types, section 3.1 (Equational constraints for C\#) of \cite{gadts}
describes this extension. This would allow a generic type to be declared
equal to another type, this would be checked statically at compile
time. For example a list flatten method could check that the list
was a list of lists by the addition of the \texttt{where T=List<U>}
clause.

\begin{lstlisting}[keywordstyle={\color{blue}},language=sharpc,tabsize=4]
public abstract class List<T> {
	...
	public abstract List<T> Append(List<T> list);
	public abstract List<U> Flatten<U>() where T=List<U>;
}

public class Nil<T> : List<T> {
	public override List<U> Flatten<U>() {
		return new Nil<U>;
	}
}

public class Cons<T> : List<T> {	
	T head; List<T> tail;
	public override List<U> Flatten<U>() {
		return this.head.Append(this.tail.Flatten());
	}
}
\end{lstlisting}


Calling \texttt{Flatten} on a \texttt{List<T>} would statically check
that \texttt{T=List<U>} where \texttt{U} is any type. Thus in the
method body of flatten we can assume that the type of \texttt{head}
is \texttt{List<U>} which has an \texttt{Append} method. While the
paper suggests this as a C\# extension generic constraints are currently
encoded at the CLI level and so we could add this as a CLI extension,
thus allowing this to be added to C\# and other languages easily. 


\section{Example}

The following shows a minimal list example, in both C\# and CIL.

\begin{lstlisting}[caption={Type equality constraints in extended C\#},keywordstyle={\color{blue}},language=sharpc,tabsize=4]
public abstract class List<T>
{
	public abstract List<T> Append(List<T> list);
	public abstract List<U> Flatten<U>() where T=List<U>;
}

public class Nil<T> : List<T>
{
	public override List<T> Append(List<T> list)
	{
		return list;
	}
	public abstract List<U> Flatten<U>()
	{
		return new Nil<U>();
	}
}

public class Cons<T> : List<T>
{
	T Head;
	List<T> Tail;

	public Cons(T head, List<T> tail) 
	{
		Head = head;
		Tail = tail;
	}

	public override List<T> Append(List<T> list)
	{
		return new Cons<T>(Head, Tail.Append(list));
	}

	public override List<U> Flatten<U>()
	{
		return Head.Append(Tail.Flatten<U>());
	}
}
\end{lstlisting}


\begin{lstlisting}[caption={Corresponding CIL},tabsize=4]
.class public abstract auto ansi beforefieldinit List<T>
extends [mscorlib]System.Object 
{
	.method family hidebysig specialname rtspecialname instance void .ctor() 
		cil managed
	{
		.maxstack 8
		ldarg.0
		call instance void [mscorlib]System.Object::.ctor()
		ret
	}

	.method public hidebysig newslot abstract virtual instance class 
		Test.List`1<!T> Append(class Test.List`1<!T> list) cil managed     
	{     }

	.method public hidebysig newslot abstract virtual instance class 
		Test.List`1<!!U> Flatten<= T List<!!0> U>() cil managed
	{     }
}

.class public auto ansi beforefieldinit Nil<T>     
	extends Test.List`1<!T> 
{     
	.method public hidebysig specialname rtspecialname instance void .ctor() 
		cil managed     
	{
		.maxstack 8
		ldarg.0
		call instance void Test.List`1<!T>::.ctor()
		ret
	}

	.method public hidebysig virtual instance class 
		Test.List`1<!T> Append(class Test.List`1<!T> list) cil managed     
	{
		.maxstack 1
		ldarg.1
		ret
	}

	.method public hidebysig virtual instance class 
		Test.List`1<!!U> Flatten<= T List<!!0> U>() cil managed
	{
		.maxstack 1
		newobj instance void Test.Nil`1<!!U>::.ctor()
		ret
	}
}

.class public auto ansi beforefieldinit Cons<T>     
extends Test.List`1<!T> 
{
	.method public hidebysig specialname rtspecialname instance void 
		.ctor(!T head, class Test.List`1<!T> tail) cil managed     
	{         
		.maxstack 2        
		ldarg.0
		call instance void Test.List`1<!T>::.ctor()
		ldarg.0
		ldarg.1
		stfld !0 Test.Cons`1<!T>::Head
		ldarg.0
		ldarg.2
		stfld class Test.List`1<!0> Test.Cons`1<!T>::Tail
		ret
	}
	
	.method public hidebysig virtual instance class 
		Test.List`1<!T> Append(class Test.List`1<!T> list) cil managed     
	{         
		.maxstack 3
		ldarg.0
		ldfld !0 Test.Cons`1<!T>::Head
		ldarg.0
		ldfld class Test.List`1<!0> Test.Cons`1<!T>::Tail
		ldarg.1
		callvirt instance class Test.List`1<!0> 
			Test.List`1<!T>::Append(class Test.List`1<!0>)
		newobj instance void Test.Cons`1<!T>::.ctor(!0, class Test.List`1<!0>)
		ret
	}

    .method public hidebysig virtual instance class
		Test.List`1<!!U> Flatten<= T List<!!0> U>() cil managed     
	{         
		.maxstack 2
		nop
		ldarg.0
		ldfld !0 Test.Cons`1<!T>::Head
		ldarg.0
		ldfld class Test.List`1<!0> Test.Cons`1<!T>::Tail
		callvirt instance class Test.List`1<!!0> 
			Test.List`1<!T>::Flatten<!!U>()
		callvirt instance class Test.List`1<!0> 
			Test.List`1<!!U>::Append(class Test.List`1<!0>)
		ret
	}

    .field private !T Head
    .field private class Test.List`1<!T> Tail
}
\end{lstlisting}


The syntax here is purely to demonstrate the intuition of the feature.
Exact syntax will be expanded on as we explore how this can be added
to the CLI specification.


\chapter{Specification changes}


\section{Generic constraints}

Section II.9.11 (Constraints on generic parameters)\cite{ecma-335} specifies
generic constraints. A type parameter that has been constrained must
be instantiated with an argument that is assignable to each of declared
constraints, and that satisfies all special constraints.

The special constraints are, \texttt{+}, \texttt{-}, \texttt{class},
\texttt{valuetype} and \texttt{.ctor}. \texttt{class} constrains the
argument to be a reference type. \texttt{valuetype} constrains the
argument to be a value type, except for any instance of\texttt{ System.Nullable<T>}.
\texttt{.ctor} constrains the argument to a type that has a public
default constructor (implicitly this means all value types as value
types always have a public default constructor). Finally \texttt{+}
and \texttt{-} are used to denote the parameter is covariant or contravariant
respectively.

While it might seem that this is a good place to add our extension
type equality constraints are a constraint on the entire parameter
list, not on each individual parameter. There's also the potential
to add an equality constraint to a non-generic method.

\begin{lstlisting}[keywordstyle={\color{blue}},language=sharpc]
class Foo<T>
{
	public void Bar(List<int> list) where T = int
	{
		...
	}
	
	public void Baz(List<string> list) where T = string
	{
		...
	}
}
\end{lstlisting}


In this example \texttt{Bar} can only be called if \texttt{Foo<T>}
was initialized with \texttt{int}, and \texttt{Baz} only if it was
initialized with \texttt{string}. A similar thing can be done with
non-generic inner types. So we need to look to add this new syntax
somewhere seperate to the generic parameter list. Preferably it would
have similar syntax for both methods and types (as generic parameters
look the same on a type declaration or method declaration). A type
declaration currently follows the pattern ``\texttt{.class ClassAttr{*}
Id {[}'<' GenPars '>'{]} {[}extends TypeSpec {[} implements TypeSpec{]}
{[}',' TypeSpec{]}{*}{]}}'', while method declarations follow the
pattern ``\texttt{.method MethAttr{*} {[}CallConv{]} Type {[}marshal
'(' {[}NativeType{]} ')'{]} MethodName {[}'<' GenPars '>'{]} '(' Parameters
')' ImplAttr{*}}''.

Adding a new clause ``\texttt{where {[}Type '=' Type{[}',' Type '=' Type{]}{*}{]}}
to method declarations after the parameter list gives us a list of Types that must 
be equal to other types. It's not strictly necceassry to have this clause on type 
declarations as for top level types it makes very little sense and for inner types 
it can be emulated be adding the clause to each method.

\section{Assignment compatability}

Assigment compatability is defined in section I.8.7 of \cite{ecma-335}, further to this 
verification assigment compatability is defined in III.1.8.1.2.3. Verification assgiment 
compataiblity is mostly defined in terms of general assigment compatability from I.8.7.3.

Verification assigment compatability is used be the verifier for determining if method calls, field 
referances and loads and stores are valid for a given type and signature. If verification assingment compatability 
is extended to understand type eqaulity constraints then operations that were unverifiable but type correct can now
be checked as verifiable as well.

Adding another rule to \emph{verifier-assignable-to} to use the rules for equality constraints is all that is needed
to enhance this part of the system.

\textbullet{} T is \emph{equal-to} U.

\subsection{\emph{equal-to}}

\emph{equal-to} is used to determine if two type names refer to the same actual type.
It uses the both the global typing enviroment $\Gamma$ and the
equality constraints on the current method $\epsilon$ which defines equal types.

\begin{prooftree}
\AxiomC{$T=U \in \epsilon$}
\LeftLabel{eq-hyp}
\UnaryInfC{$\Gamma,\epsilon \vdash T=U$}
\end{prooftree}

\begin{prooftree}
\AxiomC{$T=U \in \epsilon$}
\AxiomC{$\Gamma \vdash C<T> \,\mathit{ok}$}
\LeftLabel{eq-con}
\BinaryInfC{$\Gamma,\epsilon \vdash C<T>=C<U>$}
\end{prooftree}

\begin{prooftree}
\AxiomC{$\Gamma,\epsilon \vdash C<T>=C<U>$}
\LeftLabel{eq-decon}
\UnaryInfC{$\Gamma,\epsilon \vdash T=U$}
\end{prooftree}

\begin{prooftree}
\AxiomC{$\Gamma \vdash T \,\mathit{ok}$}
\LeftLabel{eq-refl}
\UnaryInfC{$\Gamma,\epsilon \vdash T=T$}
\end{prooftree}

\begin{prooftree}
\AxiomC{$U=T \in \epsilon$}
\LeftLabel{eq-sym}
\UnaryInfC{$\Gamma,\epsilon \vdash T=U$}
\end{prooftree}

\begin{prooftree}
\AxiomC{$\Gamma,\epsilon \vdash T=U$}
\AxiomC{$\Gamma,\epsilon \vdash U=V$}
\LeftLabel{eq-sym}
\BinaryInfC{$\Gamma,\epsilon \vdash T=V$}
\end{prooftree}

\section{Method calls}

Any method calls must be checked that any equality constraints on the method being called 
can be satisifed by the current context. This is done by checking that for each constraint listed on the 
method ($T=U$) T and U are \emph{equal-to} in the current context after applying substitution for generic parameters.

\begin{prooftree}
\AxiomC{$\Gamma \vdash m : <\bar{U}> \,\mathit{where}\, \epsilon'$}
\AxiomC{$\forall e \in (\epsilon'[\bar{T}/\bar{U}]).\Gamma, \epsilon \vdash e$}
\LeftLabel{call}
\BinaryInfC{$\Gamma,\epsilon \vdash m<\bar{T}>$}
\end{prooftree}

This is an addition to the verifiabilty rules for the call, calli and callvirt instructions.

\section{Metadata tables}

Metadata tables are specified in section 22 (Metadata logical format:
tables). We're mostly intreasted in section 22.21 (GenericParamConstraint),
which explains how generic constraints are stored in the assembly. 
\begin{quotation}
The GenericParamConstraint table has the following columns:

\textbullet{} Owner (an index into the GenericParam table, specifying
to which generic parameter this row refers) 

\textbullet{} Constraint (an index into the TypeDef, TypeRef, or TypeSpec
tables, specifying from which class this generic parameter is constrained
to derive; or which interface this generic parameter is constrained
to implement; more precisely, a TypeDefOrRef (�24.2.6) coded index) 

The GenericParamConstraint table records the constraints for each
generic parameter. Each generic parameter can be constrained to derive
from zero or one class. Each generic parameter can be constrained
to implement zero or more interfaces. 

Conceptually, each row in the GenericParamConstraint table is owned
by a row in the GenericParam table. 

All rows in the GenericParamConstraint table for a given Owner shall
refer to distinct constraints.
\end{quotation}
We need a similar table for our equality constraints. It will need
an owner (a type or method) and then two types that are constrained
to be equal within the owner. Another column that isn't necessary
but could be useful is a flags field describing the relationship between
the two types, currently this would always be set to equal, however
the system could be extended at a later date to allow less than and
greater than relationships as well.

So the TypeRelationshipConstraint table has the following columns:
\begin{itemize}
\item Owner (an index into the MethodDef table, specifying the
Method to which this constraint applies)
\item Flags (a 1-byte flag bitmask currently always set to 0. To be used
for extensions)
\item ConstraintA (an index into the TypeDef, TypeRef, or TypeSpec tables,
specifying the first type; more precisely, a TypeDefOrRef (�24.2.6)
coded index)
\item ConstraintB (an index into the TypeDef, TypeRef, or TypeSpec tables,
specifying the second type; more precisely, a TypeDefOrRef (�24.2.6)
coded index)
\end{itemize}

\chapter{Implementation changes}

\section{Runtime}

The runtime needs to be changed to load and understand equality constraint metadata, 
these changes will be focused in mono/metadata/metadata.h/c and mono/metadata/reflection.h/c.
The TypeRelationshipConstraint table is very similar to the GenericParamConstraint table, the current code
for GenericParamConstraint can guide the new code for TypeRelationshipConstraint.

Verification is done in verify.c. Additions need to be made to verifiy that type relationship constraints
are met when calling methods and to extend assingment compatability to use those type relationship constraints.

\section{corlib}

The core library needs to be changed to support type equality. Specificly the reflection libary defined in
mcs/class/corlib/System.Reflection\[.Emit\]. This API is predomintly implement via internal calls that are 
then defined in mono/metadata/icall-def.h.

\section{Cecil and IKVM}

Mono.Cecil and IKVM are well known libaries for working with CIL code. They are generally considered better
than the core library System.Reflection and System.Reflection.Emit. Any extensions of the system would be well
served by extending these libaries to understand the extension as well.

\chapter{References}

\bibliographystyle{plain}
\bibliography{typeEquality}

\end{document}
