
\chapter{Project plan}

As already briefly mentioned in the introduction this project can
be split into three major parts, investigation, design and implementation.


\section{Investigation}

The first part of the project is an investigation into value dependent
types and similar systems. This has already been covered in our background
research. From this we have an understanding of how these systems
are useful and how they can be designed and implemented and this will
guide us on the design of the CLI extension. This investigation forms
the background research part of the final report and has mostly been
done as part of this interim report.


\section{Design and implementation}

The second part of the project is to design and implement extensions
to the CLI. For each extension we will show what changes need to be
made to ECMA-335 to support the extension, and implement the extension
in the open source CLI Mono. We list each of these and propose some
detail of it's implementation, all syntax is highly subject to change
however.
\begin{enumerate}
\item GADT extensions - 25\textsuperscript{th} February
\item Primitive value parameters - 25\textsuperscript{th} March
\item Value parameter constraints and operations - 22\textsuperscript{nd}
April
\item User defined types as value parameters - 20\textsuperscript{th} May
\item Further improvements and formalization - 17\textsuperscript{th} June
\item Final report - 24\textsuperscript{th} June
\end{enumerate}

\subsection{GADTs}

The first extension will be to add type equality constraints and a
match and bind instruction to the CLI. To do this we will take the
ideas from \cite{gadts} and translate them to apply to the CLI.

\begin{lstlisting}[caption={Type equality constraints in extended C\#\protect \\
Extension of listing \ref{lis:csharp gadt}},escapechar={~},keywordstyle={\color{blue}},language=sharpc]
public abstract class List<T>
{
	public abstract List<T> Append(List<T> list);
	public abstract List<U> Flatten<U>() ~\colorbox{yellow}{where T=List<U>;}~
}

public class Nil<T> : List<T>
{
	public override List<T> Append(List<T> list)
	{
		return list;
	}
	public abstract List<U> Flatten<U>() // type constraints are inherited
	{
		return new Nil<U>();
	}
}

public class Cons<T> : List<T>
{
	T Head;
	List<T> Tail;

	public Cons(T head, List<T> tail) 
	{
		Head = head;
		Tail = tail;
	}

	public override List<T> Append(List<T> list)
	{
		return new Cons<T>(Head, Tail.Append(list));
	}

	public override List<U> Flatten<U>() // type constraints are inherited
	{
		return Head.Append(Tail.Flatten<U>()); // invalid in standard C#
	}
}
\end{lstlisting}


\begin{lstlisting}[caption={Corresponding CIL},escapechar={~},language=cil]
.class public abstract auto ansi beforefieldinit List<T>
extends [mscorlib]System.Object 
{
	.method family hidebysig specialname rtspecialname instance void .ctor() 
		cil managed
	{
		.maxstack 8
		ldarg.0
		call instance void [mscorlib]System.Object::.ctor()
		ret
	}

	.method public hidebysig newslot abstract virtual instance class 
		Test.List`1<!T> Append(class Test.List`1<!T> list) cil managed     
	{     }

	.method public hidebysig newslot abstract virtual instance class 
		Test.List`1<!!U> Flatten<~\colorbox{yellow}{= T List<!!0>}~ U>() cil managed
	{     }
}

.class public auto ansi beforefieldinit Nil<T>     
	extends Test.List`1<!T> 
{     
	.method public hidebysig specialname rtspecialname instance void .ctor() 
		cil managed     
	{
		.maxstack 8
		ldarg.0
		call instance void Test.List`1<!T>::.ctor()
		ret
	}

	.method public hidebysig virtual instance class 
		Test.List`1<!T> Append(class Test.List`1<!T> list) cil managed     
	{
		.maxstack 1
		ldarg.1
		ret
	}

	.method public hidebysig virtual instance class 
		Test.List`1<!!U> Flatten<~\colorbox{yellow}{= t list<!!0>}~ U>() cil managed
	{
		.maxstack 1
		newobj instance void Test.Nil`1<!!U>::.ctor()
		ret
	}
}

.class public auto ansi beforefieldinit Cons<T>     
extends Test.List`1<!T> 
{
	.method public hidebysig specialname rtspecialname instance void 
		.ctor(!T head, class Test.List`1<!T> tail) cil managed     
	{         
		.maxstack 2        
		ldarg.0
		call instance void Test.List`1<!T>::.ctor()
		ldarg.0
		ldarg.1
		stfld !0 Test.Cons`1<!T>::Head
		ldarg.0
		ldarg.2
		stfld class Test.List`1<!0> Test.Cons`1<!T>::Tail
		ret
	}
	
	.method public hidebysig virtual instance class 
		Test.List`1<!T> Append(class Test.List`1<!T> list) cil managed     
	{         
		.maxstack 3
		ldarg.0
		ldfld !0 Test.Cons`1<!T>::Head
		ldarg.0
		ldfld class Test.List`1<!0> Test.Cons`1<!T>::Tail
		ldarg.1
		callvirt instance class Test.List`1<!0> 
			Test.List`1<!T>::Append(class Test.List`1<!0>)
		newobj instance void Test.Cons`1<!T>::.ctor(!0, class Test.List`1<!0>)
		ret
	}

    .method public hidebysig virtual instance class
		Test.List`1<!!U> Flatten<~\colorbox{yellow}{= T List<!!0>}~ U>() cil managed     
	{         
		.maxstack 2
		nop
		ldarg.0
		ldfld !0 Test.Cons`1<!T>::Head
		ldarg.0
		ldfld class Test.List`1<!0> Test.Cons`1<!T>::Tail
		callvirt instance class Test.List`1<!!0> 
			Test.List`1<!T>::Flatten<!!U>()

		// the following callvirt would not verify in the standard CLI

		callvirt instance class Test.List`1<!0> 
			Test.List`1<!!U>::Append(class Test.List`1<!0>) 
		ret
	}

    .field private !T Head
    .field private class Test.List`1<!T> Tail
}
\end{lstlisting}


As per the style used in ECMA-335, match would be specified something
like the following.

\begin{tabular}{|c|c|>{\centering}p{0.75\linewidth}|}
\hline 
\multicolumn{3}{|c}{match}\tabularnewline
\hline 
\hline 
Format & Assembly Format & Description\tabularnewline
\hline 
\ldots{} & match typetoken & matches and object against an open or closed generic type and returns
the object cast to that type and the RuntimeHandles for the types
required for closure if typetoken is an open generic type.\tabularnewline
\hline 
\end{tabular}

Stack Transition: \ldots{}, obj $\rightarrow$ \ldots{}, obj, RuntimeHandles


\subsection{Value parameters}

The second extension is to add primitive value parameters to types
and methods. This extension will only allow primitive types as parameters
and has no support for constraints or operations on parameters at
compile time.

\begin{lstlisting}[caption={Value parameters in extended C\#},keywordstyle={\color{blue}},language=sharpc]
public class Value<int value>
{
	public static int Dup<int v>(Value<v> a, Value<v> b)
	{
		return Value<v>.value * 2;
	}

	public static void Print()
	{
		System.Console.WriteLine(value);
	}
}
\end{lstlisting}


\begin{lstlisting}[caption={Corresponding CIL},language=cil]
.class public auto ansi beforefieldinit Value``1<int value>
extends [mscorlib]System.Object 
{
	.method family hidebysig specialname rtspecialname instance void .ctor() 
		cil managed
	{
		.maxstack 8
		ldarg.0
		call instance void [mscorlib]System.Object::.ctor()
		ret
	}

	.method public hidebysig static int32 
		Dup``1<int v>(class Value``1<$$0> a, class Value``1<$$0> b) cil managed
	{
		.maxstack 2
		.locals init (int32 temp)
		ldsfld int32 Value``1<$$0>::value
		ldc.i4.2 
		mul
		ret
	}

	.method public hidebysig static void Print() cil managed
	{
		ldsfld int32 Value``1<$0>::value
		call void [mscorlib]System.Console::WriteLine(int32)
		ret
	}
}
\end{lstlisting}



\subsection{Value constraints and operations}

The third extension is to add constraints and operations to value
parameters. It's unclear at this stage quite how this could work in
CIL and so we don't provide an example of it.


\subsection{User defined types as value parameters}

The final extension is to allow value parameters to be values of user
defined types, not just primitive types.

\begin{lstlisting}[caption={User defined types as value parameters in extended C\#},keywordstyle={\color{blue}},language=sharpc]
public class Value<MyClass c>
{
	public static int Access<MyStruct s>()
	{
		return s.field;
	}

	public static void Print()
	{
		System.Console.WriteLine(c.property);
	}
}
\end{lstlisting}


\begin{lstlisting}[caption={Corresponding CIL},language=cil]
.class public auto ansi beforefieldinit Value``1<class MyClass c>
extends [mscorlib]System.Object 
{
	.method family hidebysig specialname rtspecialname instance void .ctor() 
		cil managed
	{
		.maxstack 8
		ldarg.0
		call instance void [mscorlib]System.Object::.ctor()
		ret
	}

	.method public hidebysig static int32 
		Access``1<MyStruct s>() cil managed
	{
		.maxstack 2
		ldvalue $$0
		ldfld int32 MyStruct::field
		ret
	}

	.method public hidebysig static void Print() cil managed
	{
		ldvalue $0
		callvirt instance int32 MyClass::get_property()    
		call void [mscorlib]System.Console::WriteLine(int32)
		ret
	}
}
\end{lstlisting}

\chapter{Testing}

As part of our implementation we will develop a number of tests to show that
our system works. These tests will show that there is no CLI code that does
not run in the new system (backwards compatibility) and that the new system is
correct.

Normally these tests would be written in CIL assembly, however writing tests in
CIL assembly would be time consuming and error prone. Instead test writing will
be assisted by an assembly rewriter. The rewriter will look for calls to
special methods and attributes in a CLI assembly and rewrite the bytecode and
metadata to use features in our extended system.  Assembly rewriting will be
done using the CLI metadata reflection, writing and rewriting via the
\texttt{System.Reflection} namespace.  Our implementation will extend these
objects to support our new metadata constructs. With this we can mark up
standard C\# or F\# code with special methods and attributes and rewrite the
resulting assemblies to include our metadata.

This system of testing has a number of advantages. Firstly it allows us to
write our tests in a high level language such as C\# instead of low level CIL.
The main benefit of writing in a high level language should be clear, it's much
easier. However we also get another less obvious benefit, that we are able to
test both imperative (C\#) and functional (F\#) code. This should mean coverage
over most aspects of the CLI.

Secondly we can run the tests on the standard CLI, this gives us something to
test against. Running in the standard CLI is not equivalent to running the
rewritten program in the new CLI (which we will call CLI+), but does allow some
deductions to be made.

\section{Type equality}

For type equality there is two methods we need to define. \texttt{U Cast<T,
U>(T obj)} that checks \texttt{T} and \texttt{U} are equal at runtime and casts
\texttt{obj} to type \texttt{U} if they are; and \texttt{void EqualTypes<T,
U>()} that will check that T and U are equal types at runtime. Both these
methods will throw an exception \texttt{TypeEqualityException} if \texttt{T}
and \texttt{U} are not equal.

The rewriter will search an assembly for uses of \texttt{Cast<T, U>} and
\texttt{EqualTypes<T,U>}. Any use of \texttt{EqualTypes<T,U>} will be removed
and the method metadata rewritten to include the new type equality tags. Any
use of \texttt{Cast<T, U>} will also rewrite the metadata to include the new
type equality tags and will also remove the call to \texttt{Cast}.

\begin{lstlisting}[caption={Cast},keywordstyle={\color{blue}},language=sharpc]
public static class TypeEquality
{
	public static U Cast<T, U>(T obj)
	{
		if(typeof(T) == typeof(U))
    {
			return (U)(Object)obj;
		}
		else
		{
			throw new TypeEqualityException();
		}
	}
	public static void EqualTypes<T, U>()
	{
		if(typeof(T) != typeof(U))
		{
			throw new TypeEqualityException();
		}
	}
}
\end{lstlisting}


A program $P$ is the source code in a CLI language, such as C\#. It may or may
not make use of \texttt{TypeEquality.Cast} and
\texttt{TypeEquality.EqualTypes}. It can be compiled by a compiler $C$ to give
a CIL assembly, this assembly can be run on a runtime to give a value. The two
runtimes are $CLI$ and $CLI+$, values are either some value $v$ representing
the overall act of computation done by the program or an exception thrown by
the program, the only exceptions we care about are $TypeEqualityException$, or
$VerificationException$.  Finally a CIL assembly can be rewritten by the
rewriter $R$ to produce a CIL+ assembly.

If $R$ and CLI+ are correct then the following statements should hold:
\begin{lemma}
$C(P) \underset{CLI}{\rightarrow} TypeEqualityException \implies R(C(P))
\underset{CLI+}{\rightarrow} VerificationException$
\end{lemma}
\begin{lemma}
$R(C(P)) \underset{CLI+}{\not\rightarrow} VerificationException \implies
C(P) \underset{CLI}{\not\rightarrow} TypeEqualityException$
\end{lemma}
\begin{lemma}
$C(P) \underset{CLI}{\rightarrow} v \wedge R(C(P))
\underset{CLI+}{\not\rightarrow} VerificationException \implies R(C(P))
\underset{CLI+}{\rightarrow} v$
\end{lemma}
\begin{lemma}
$C(P) \underset{CLI}{\rightarrow} v \iff C(P) \underset{CLI+}{\rightarrow} v$
\end{lemma}

Lemma 1 states that if a program throws a \texttt{TypeEqualityException} in
the standard CLI then it will throw a \texttt{VerificationException} in the new
CLI. If CLI+ doesn't throw a \texttt{VerificationException} then we know
something is wrong with either the rewriter or the new runtime. For the sake of
these tests we will assume that the rewriter is correct. This property does not
hold in reverse ($R(C(P)) \underset{CLI+}{\rightarrow} VerificationException
\implies C(P) \underset{CLI}{\rightarrow} TypeEqualityException$) as the call
to \texttt{Cast<T, U>} or \texttt{EqualTypes<T, U>} might not be hit be every
control flow path.

Lemma 2 states that if the rewritten program runs in CLI+ and does not throw a
\texttt{VerificationException} then running the code in the CLI will not throw
a \texttt{TypeEqualityException}. If a \texttt{TypeEqualityException} is thrown
then something is wrong with the new runtime.

Lemma 3 states that if the program computes a value $v$ in the standard CLI
and verifies correctly in the new CLI then the new CLI should compute the same
value $v$.

Lemma 4 states that if the same CLI assembly (no rewriting) is run on both
runtimes they should compute the same value. While this is similar to lemma 3
it's difference is the lack of any assembly rewriting. 

All this together means that we can do some verification of our new system
against the old system. If a program ran correctly in the old system it should
also run correctly in the new system (lemma 4). If a program has correct equality type
constraints then it should run in the old system (and by the lemma 4 also
run in the new system) and when rewritten it should run in the new system
(lemma 3). If a program throws a \texttt{TypeEqualityException} then it should
throw a \texttt{VerificationException} when rewritten and run in the new
system.


\chapter{Evaluation plan}


\section{Semantics}

Defining semantics for what value dependence means can be done in
two ways. Firstly we could extend the ECMA specification (\cite{ecma-335}).
Secondly we could take a formal specification of the CLI and extend
that. While work has been done on formalization of CLI languages such
as C\#, work on formalizing the CLI does not seem to have been done.
If we can work out how value dependence should work in the CLI then
extending the ECMA specification is a required aim, as it is the basis
for compiler writers targeting the CLI. Extending a formal specification
would be a stretch goal to complete once other goals are achieved,
both because it may require translating our extension to C\# to use
a lightweight C\# formalization based on featherweight GJ and secondly
as a formalization is not required for the implementation work.


\section{Implementation}

Given an extension to the CLI specification we want to show that the
extension can be implemented. To do this we will extend the open source
Mono run time to support value dependent types. The most important
aspect is correctness but performance should be kept in mind. As pointed
out in section \ref{sub:Performance} we want certain performance
characteristics out of the system. For each extension we will include
test cases to demonstrate that the implementation is correct. Test
cases will be created both during the development of each extension
and, time permiting, some time after the extension is finished.