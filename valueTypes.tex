\chapter{Values as type parameters}

Our second extension is the addition of values as type parameters to the CLI.
Drawing on what we've learnt in our background research, our knowledge of the 
CLI and the design goals laid out by out motvating example we discuss some of
our ideas and issues with value dependent types.

\subsection{Concept of equality}

To be able to say if a type \texttt{T} that depends on a value of
\texttt{U} (that is \texttt{T<U a>}) is equal to \texttt{T<U b>} requires
us to say what it means for \texttt{a} and \texttt{b} to be equal.
All CLI objects have a method \texttt{bool Equal(object obj)} which
we could use, however this is clearly unsound as an implementation
of \texttt{Equal} could return nonsense, or never return at all.

Instead we propose using structural equality. That is
two values of any reference type are equal if and only if they are
the same reference, and two values of a value type are equal if and
only if all their fields are equal in this manner as well. 

In practice this amounts to checking that the values have the same bytes in memory.
At runtime this is an easy check to make and dynamic type checks and reflection would have
no issues with it. Doing this at compile time requires that each instantiation of a value 
parameter is constant, for value types this is simple to achieve while still being easily usable.
Two separate instantiations of a type \texttt{T<int i>} with the same \texttt{int} value will both end 
up with the same byte values for the \texttt{int} value (although how and where this value is stored 
still needs to be discussed). For custom value types some extra work is required to make sure they are
immutable but instantiation remains simple. 

However given some reference type \texttt{U} with a constructor \texttt{U(int i)} two separate instantiations of 
a type \texttt{T<U u>} done via two \texttt{new} expressions will result in two distinct references. 
Given that all these values must be constant at compile time it's not clear how this could work. We can't 
reference the value from another type specification as that itself requires the reference.

\begin{lstlisting}[caption={An issue with reference values},keywordstyle={\color{blue}},language=sharpc]
var myT = new T<new U(1)>();
var myOtherT = new T<T<new U(1)>.u>();
}
\end{lstlisting}

We could extend the definition of \texttt{literal}(\texttt{const} in C\#) to allow constants of custom types this would allow us to 
initialize and store the reference once and then use it in type parameters.

\begin{lstlisting}[caption={Literal references},keywordstyle={\color{blue}},language=sharpc]
const U U1 = new U(1);

public T<U1> SomeMethod(T<U1> t)
{
	return t;
}
\end{lstlisting}

While this seems ideal due to the way \texttt{literal} works this would cause significant difficulty for reference types,
but it would work for value types and would solve the storage issue mentioned earlier, so still a wanted extension.
The reason for the difficulty is that uses of \texttt{literal} fields replace the use site with the value at compile time. 
The value for a reference is a managed reference pointer, deciding where that pointer should point is impossible at compile time

\subsection{Immutability}

Type preservation means that an expression's type should not change
under evaluation, therefore value type parameters should be immutable.
As shown in subsection \ref{sub:Literal-and-initonly} the CLI does
not have strong support for immutability. As such our initial work
will concentrate on using the primitive types as their immutability
can be easily guaranteed.

As noted above the values used as parameters should be immutable, allowing these values to change at runtime would be unsound 
as the system has already made judgements based on the static value. For primitive types immutability is easily achieved by making 
it impossible to write to or take the address of the parameter. This also covers us for custom value types, as although instance 
methods on value types take a managed pointer to the value (which could then be used to write to the fields) there is no way to
get this reference in the first place. The value must be copied to a local variable and then instance methods can be called on that.

For references the value we care about is the reference itself, the object pointed to does not need to be immutable.

\subsection{Operations}

Once we have the ability to mark up types with values, we will want
to use the value for operations. Either for changing the value before
passing it on as another type parameter or for using at runtime. Support
for using the value in normal methods at runtime seems trivial, just
expose it similar to a static readonly field. However supporting the
ability to do operations on value parameters before passing them to
another type constructor is more challenging. Firstly it will require
some effort to fit into the CIL bytecode, currently opcodes are only
allowed in method bodies, we would have to either point to a method
to calculate the operations on value parameters or find some other
way to fit opcodes at the declaration level. Secondly we have the
issue of soundness as user defined operations can do anything.