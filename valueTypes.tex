\chapter{Values as type parameters}

Our second extension is the addition of values as type parameters to the CLI.
Drawing on what we've learnt in our background research, our knowledge of the 
CLI and the design goals laid out by our motivating example, we discuss some of
our ideas and issues with value dependent types.

These ideas will be often discussed in terms of high level design, but 
must always take into account that they are really targeted at the CLI level.

\subsection{Concept of equality}

To be able to say if a type \texttt{T} that depends on a value of
\texttt{U} (that is \texttt{T<U a>}) is equal to \texttt{T<U b>} requires
us to say what it means for \texttt{a} and \texttt{b} to be equal.
All CLI objects have a method \texttt{bool Equal(object obj)} which
we could use, however this is clearly unsound as an implementation
of \texttt{Equal} could return nonsense, or never return at all.

Instead, we propose using structural equality. That is,
two values of any reference type are equal if and only if they are
the same reference, and two values of a value type are equal if and
only if all their fields are equal in this manner as well. 

In practice, this amounts to checking that the values have the same bytes in memory.
At runtime this is an easy check to make and dynamic type checks and reflection would have
no issues with it. Doing this at compile time requires that each instantiation of a value 
parameter is constant. For value types this is simple to achieve while still being easily usable.
Two separate instantiations of a type \texttt{T<int i>} with the same \texttt{int} value will both end 
up with the same byte values for the \texttt{int} value (although how and where this value is stored 
still needs to be discussed). For custom value types some extra work is required to make sure they are
immutable, but instantiation remains simple. 

However, given some reference type \texttt{U} with a constructor \texttt{U(int i)} two separate instantiations of 
a type \texttt{T<U u>} done via two \texttt{new} expressions will result in two distinct references. 
Listing \ref{lst:refissue} demonstrates this issue. The parameter passed to the inner \texttt{T} constructor of the second 
line is a different instance of \texttt{U} to that present in the first line.
Given that all these values must be constant at compile time its not clear how this could work. We can't 
reference the value from another type specification as that itself requires the reference.

\begin{lstlisting}[label={lst:refissue},caption={An issue with reference values},keywordstyle={\color{blue}},language=sharpc]
var myT = new T<new U(1)>();
var myOtherT = new T<T<new U(1)>.u>();
}
\end{lstlisting}

We could extend the definition of \texttt{literal}(\texttt{const} in C\#) to allow constants of custom types. This would allow us to 
initialize and store the reference once and then use it in type parameters.

\begin{lstlisting}[caption={Literal references},keywordstyle={\color{blue}},language=sharpc]
const U U1 = new U(1);

public T<U1> SomeMethod(T<U1> t)
{
	return t;
}
\end{lstlisting}

While allowing any type as a constant seems ideal, due to the way \texttt{literal} works this would cause significant 
difficulty for reference types, but it would work for value types and would solve the storage issue mentioned earlier, 
so still a wanted extension.
The reason for the difficulty is that uses of \texttt{literal} fields replace the use site with the value at compile time. 
The value for a reference is a managed reference pointer, deciding where that pointer should point is impossible at compile time

\subsection{Immutability}

Type preservation means that an expression's type should not change
under evaluation, therefore value type parameters should be immutable.
As shown in subsection \ref{sub:Literal-and-initonly}, the CLI does
not have strong support for immutability. As such, our initial work
will concentrate on using the primitive types as their immutability
can be easily guaranteed.

As noted above, the values used as parameters should be immutable, allowing these values to change at runtime would be unsound 
as the system has already made judgements based on the static value. For primitive types immutability is easily achieved by making 
it impossible to write to or take the address of the parameter. This also covers us for custom value types, as although instance 
methods on value types take a managed pointer to the value (which could then be used to write to the fields) there is no way to
get this reference in the first place. The value must be copied to a local variable and then instance methods can be called on that.

For references the value we care about is the reference itself, the object pointed to does not need to be immutable.

\subsection{Operations}

Once we have the ability to mark up types with values, we will want
to use the value for operations. Either for changing the value before
passing it on as another type parameter or for using at runtime. 
For example to be able to concatenate two fixed arrays together as shown in listing \ref{lst:arrayconcat}.

\begin{lstlisting}[label={lst:arrayconcat},caption={Concatenate two arrays},keywordstyle={\color{blue}},language=sharpc]
public class Array<T, int length> { 
	...
	
	public Array<T, length + n> Concatenate<int n>(Array<T, n> other) { ... }
}
\end{lstlisting}

Support for using the value in normal methods at runtime seems trivial, just
expose it similar to a static readonly field. However, supporting the
ability to do operations on value parameters before passing them to
another type constructor is more challenging. Firstly, it will require
some effort to fit into the CIL bytecode, currently opcodes are only
allowed in method bodies, we would have to either point to a method
to calculate the operations on value parameters or find some other
way to fit opcodes at the declaration level. Secondly, we have the
issue of soundness as user defined operations can do anything.

Another soloution to supporting operations would be to define a mini expression langaguge. This might 
support just the basic opertations such as \texttt{+}, \texttt{*} and \texttt{<<}. Unfortuantly about
the only problem this solves is unsoundness (as we can restrict the set of operations allowed). Fitting 
this into the metadata is still an issue, and now we've introduced the problem of having to define this 
language, which is pretty much just a mapping down to CIL bytecodes anyway!

Operations on values exasperates the storage issues again. While above we described using the constant table
metadata to store values and paramaters would just point to that table we can no longer do this. If values can be 
changed then we are no longer working with constants, and we can't solve this problem in the same was as C++ templates 
(expanding and running all template operations at compile time) because we have to support dynamic loading and linking.

If our \texttt{Array} type from above was in a library then we need to record in the metadata that the return type
of \texttt{Concatenate} is \texttt{length + n}. We have no way of collapsing that to a single constant as \texttt{Array}
has not yet been instantiated.

\begin{lstlisting}[caption={Using Array},keywordstyle={\color{blue}},language=sharpc]
Array<int, 4> arr1 = new Array<int, 4>(new int[] { 1, 2, 3, 4});
Array<int, 2> arr2 = new Array<int, 2>(new int[] { 5, 6});
Array<int, 6> arr3 = arr1.Concatenate<2>(arr2); 
// in all likelihood the <2> would be infered by high level languges
\end{lstlisting}


\subsection{Variable structures}

As described in our motivating example, one use case of dependent types is for vector types.

We want these vectors to be value types with all their data allocated inline. This makes for fast and local
allocations which is good for cache coherency, but also makes it possible to easily transfer them to native
APIs such as OpenGL and OpenCL.

The standard CLI can define types with a fixed size, either via multiple fields and explicit layout or via explicit size.
Neither of these allow a types inline size to be changed per use case, the only dynamically sized types in the CLI are
arrays which are allocated on the heap.

\begin{lstlisting}[caption={Vector using CLI array}, keywordstyle={\color{blue}},language=sharpc]
public struct Vector
{
	public readonly float[] values;
	
	public Vector(int n) {
		values = new float[n];
	}
	
	...
}
\end{lstlisting}

While this allows vectors of different dimensions, it is not cache efficient, easy to copy to native APIs or immutable.
It also requires many runtime checks to make sure that the vectors passed into methods are the right size.

\begin{lstlisting}[caption={Runtime checks needed when using arrays}, keywordstyle={\color{blue}},language=sharpc]
public static float Dot(Vector a, Vector b)
{
	if(a.values.length != b.values.length)
		throw new ArgumentException("a and b are different sizes.");
		
	float dot = 0;
	for(int i=0; i<a.Length; ++i)
	{
		dot += a.values[i]*b.values[i];
	}
	return dot;
}
\end{lstlisting}

C\# allows the declaration of fixed size arrays. These are implemented via an internal, explicitly sized class and pointer
arithmetic. Due to the pointer arithmetic they are only allowed in an unsafe context (a C\# construct to delimit unsafe code).
The size of the fixed array must be a compile time constant. Given the fixed size and the generally unwanted requirement for
unsafe code, fixed arrays don't have many uses. Certainly for these small vector types (which are often accessed for a specific 
field such as \texttt{X}) it's better to use multiple fields.

\begin{lstlisting}[caption={A fixed size vector type}, keywordstyle={\color{blue}},language=sharpc]
public unsafe struct Vector3
{
	public fixed float values[3]; 
	
	...
}
\end{lstlisting}

\begin{lstlisting}[caption={Translation to CIL}, keywordstyle={\color{blue}},language=cil]
.class sequential ansi sealed nested public beforefieldinit Vector3
    extends [mscorlib]System.ValueType
{
    .field public valuetype Test.Program/Vector3/<Values>e__FixedBuffer0 Values
    {
        .custom instance void [mscorlib]System.Runtime.CompilerServices.FixedBufferAttribute::.ctor(class [mscorlib]System.Type, int32) = 
			{ type(float32) int32(3) }
    }

    .class sequential ansi sealed nested public beforefieldinit <Values>e__FixedBuffer0
        extends [mscorlib]System.ValueType
    {
        .custom instance void [mscorlib]System.Runtime.CompilerServices.UnsafeValueTypeAttribute::.ctor()
        .custom instance void [mscorlib]System.Runtime.CompilerServices.CompilerGeneratedAttribute::.ctor()
        .field public float32 FixedElementField

    }
}
\end{lstlisting}

If we introduce values as type paramters we want some way of adding or removing data based on that value. A \texttt{Vector<2>} 
should only take up 2 floats of storage. 

\begin{lstlisting}[caption={A fixed size vector type}, keywordstyle={\color{blue}},language=sharpc]
public struct Vector<int n>
{
	//fixed buffers are currently supported in C\#,
	//but their size must be a compile time constant
	public fixed float values[n]; 
	
	...
}
\end{lstlisting}

There seems two immediate ways to solve this problem. Firstly by allowing a reference to the value parameter in the 
\texttt{FixedBufferAttribute} or by adding fixed size buffers to the CLI as a supported type.