\documentclass[english]{report}
\usepackage[T1]{fontenc}
\usepackage[latin9]{inputenc}
\usepackage{listings}
\setcounter{secnumdepth}{3}
\setcounter{tocdepth}{3}
\usepackage{color}
\definecolor{note_fontcolor}{rgb}{0.80078125, 0.80078125, 0.80078125}
\usepackage{array}
\usepackage{longtable}

\makeatletter

%%%%%%%%%%%%%%%%%%%%%%%%%%%%%% User specified LaTeX commands.
\usepackage{amsmath,amssymb,dot2texi,tikz,hyperref,enumitem,amsthm}
\usetikzlibrary{shapes,arrows}

\usepackage[T1]{fontenc}
\usepackage[sc]{mathpazo}
\usepackage{fullpage}
\usepackage{bussproofs}
\usepackage{framed}
\usepackage{centernot}

\usepackage{color}

\usepackage{float}
\floatstyle{boxed}
\restylefloat{figure}

\usepackage{url}
\usepackage{listings}

\newtheorem{lemma}{Lemma}

\def\signed #1{{\leavevmode\unskip\nobreak\hfil\penalty50\hskip2em
  \hbox{}\nobreak\hfil(#1)%
  \parfillskip=0pt \finalhyphendemerits=0 \endgraf}}

\newsavebox\mybox
\newenvironment{aquote}[1]
  {\savebox\mybox{#1}\begin{quote}}
  {\signed{\usebox\mybox}\end{quote}}
  
\lstdefinelanguage{sharpc}{language=[Sharp]C}

% "define" Scala
\lstdefinelanguage{scala}{
  morekeywords={abstract,case,catch,class,def,%
    do,else,extends,false,final,finally,%
    for,if,implicit,import,match,mixin,%
    new,null,object,override,package,%
    private,protected,requires,return,sealed,%
    super,this,throw,trait,true,try,%
    type,val,var,while,with,yield},
  otherkeywords={=>,<-,<\%,<:,>:,\#,@},
  sensitive=true,
  morecomment=[l]{//},
  morecomment=[n]{/*}{*/},
  morestring=[b]",
  morestring=[b]',
  morestring=[b]"""
}

% "define" cil
\lstdefinelanguage{cil}{
  morekeywords={.assembly, .method, static, void, extern, .entrypoint,%
	.maxstack, public, private, hidebysig, class, cil, managed, .locals,%
	auto, ansi, beforefieldinit, extends, .property, instance, .get, .set,%
	.field, specialname, sealed, implements,serializable},
  morecomment=[l]{//},
  morecomment=[n]{/*}{*/},
  morestring=[b]",
  morestring=[b]',
}

\newcommand{\HRule}{\rule{\linewidth}{0.5mm}}

\usepackage{caption}
\DeclareCaptionFont{black}{\color{black}}

\DeclareCaptionFormat{listing}{%
  \parbox{\textwidth}{\fbox{\parbox{\textwidth}{#1#2#3}}\vskip4pt}}

\captionsetup[lstlisting]{format=listing,labelfont=black,textfont=black}
\lstset{tabsize=4,frame=single,xleftmargin=\fboxsep,xrightmargin=-\fboxsep}

\makeatother

\usepackage{babel}

\begin{document}
\begin{titlepage}
\begin{center}
%Upper part of the page 
\includegraphics[width=0.50\textwidth]{imperial_crest_colour.jpg}\\[1cm]
\textsc{\LARGE Imperial College London}\\[1.5cm]
\textsc{\Large Project Report}\\[0.5cm]
% Title 
\HRule \\[0.4cm] 
{ \huge \bfseries Value dependent types for the CLI}\\[0.4cm]
\HRule \\[1.5cm]
% Author and supervisor 
\begin{minipage}{0.4\textwidth} 
\begin{flushleft} \large 
\emph{Author:}\\ 
Fraser \textsc{Waters}\\
\href{mailto:fraser.waters08@imperial.ac.uk}{fraser.waters08@imperial.ac.uk} 
\end{flushleft} 
\end{minipage} 
\begin{minipage}{0.4\textwidth} 
\begin{flushright} \large 
\emph{Supervisor:} \\ Professor Sophia \textsc{Drossopoulou} 
\end{flushright} 
\end{minipage}

\vfill

% Bottom of the page 
{\large \today}

\end{center}
\end{titlepage}

\tableofcontents

\begin{abstract}
Value dependent types are a powerful extension to type systems allowing
types to be parametrized by terms. This project looks into how value
dependent types could be introduced to the CLI, the underlying virtual
machine specification for C\#, Visual Basic, F\# and many other languages,
to allow more programs to be succinctly expressed at the CLI level and 
exposed to these languages. We show how the CLI can be enhanced with type
equality constraints, and discuss some issues and solutions to allow 
values as type parameters.
\end{abstract}

\chapter{Introduction}

\section{Value dependent types}

Dependent types allow expressions to be statically typed based on values;
rather like how parametric types can be based on other types. There
are some functional languages such as Agda\cite{agda}, Coq\cite{coq},
Idris\cite{idris} and Cayanne\cite{cayenne} that support dependent
types, but in object oriented languages dependent types are not so
common. While fully general value dependent types are rare, some weaker
versions, including path dependent types (section \ref{sec:Path-dependent-types})
and virtual types (section \ref{sec:Virtual-types}), are used in
some mainstream languages. Notably Scala\cite{scala} supports both
path dependence and virtual types, F\#\cite{fsharp} supports units
of measure (section \ref{sec:Fsharp-units}) allowing numbers to be
typed based on a unit value, and C++ has templates (section \ref{sec:Cpp-templates})
that can be parametrized on values.


\section{Motivation}

One of the main motivations for looking into value dependence is their potential 
use for work in graphics and physics applications. Vectors and matrices
in these problem domain are often small (3 or 4 elements). Using types such
as Vector3, Vector4, Matrix3x4 which are 3 and 4 element float vectors
and a 3 by 4 float matrix respectively, makes writing code much easier
than working with multiple float variables. Currently there is no
nice way to represent all the different sizes for vector and matrix types
in C\# (or any other CLI language). This has lead me to the
creation of a numeric type generator, a separate program that outputs
the source code for a pre defined set of configurations (currently
Vector2 to Vector8 and Matrix2x2 up to Matrix4x4). While the use of
these types is mostly acceptable, extending them is difficult. As shown
below, using the generator requires writing the code in literal strings,
these literals can not be checked at compile time for obvious mistakes
and the IDE does not offer auto completion when writing them, the
generator has to be run then the emitted code must be compiled to
see any problems. The following shows a section of code used to generate
all the required dot product functions. \texttt{Components} is a list
of component indices 0, 1, 2 etc. \texttt{WriteLine} writes a string
to the code file using the current indent level, \texttt{Indent} and
\texttt{Dedent} increase and decrease the indent level.

\begin{lstlisting}[keywordstyle={\color{blue}},language=sharpc,showstringspaces=false]
if (!Type.IsCLSCompliant) { WriteLine("[CLSCompliant(false)]"); }             
WriteLine("public static float Dot({0} left, {0} right)", Name);             
WriteLine("{");             
Indent();
var dotproduct = string.Join("+", Components.Select(
	component => string.Format("left[{0}]*right[{0}]", component)));
WriteLine("return {0};", dotproduct);
Dedent();
WriteLine("}");
\end{lstlisting}


With value dependence we could write the code directly to be compiled,
skipping the generator step, and allowing the use of auto complete
and faster iteration times.

\begin{lstlisting}[keywordstyle={\color{blue}},language=sharpc]
public static float Dot<int n>(Vector<n> left, Vector<n> right)
{
	float dot = 0;
	for(int i=0; i<n; ++i)
	{
		dot += left[i]*right[i];
	}
	return dot;
}
\end{lstlisting}
The usage of these types would remain nearly the same, the following
shows how they look at the moment compared to what they might look
like with value dependence.

\begin{lstlisting}[caption={Current method},keywordstyle={\color{blue}},language=sharpc]
var a = new Vector3(1, 1, 1);
var b = new Vector3(2, 2, 2);

var ab = b - a;
var dot = Vector.Dot(ab, a);
\end{lstlisting}


\begin{lstlisting}[caption={Proposal},keywordstyle={\color{blue}},language=sharpc]
var a = new Vector<3>(1, 1, 1);
var b = new Vector<3>(2, 2, 2);

var ab = b - a;
var dot = Vector.Dot(ab, a);
\end{lstlisting}


As you can see there isn't a big difference. Users get more benefit
with the latter as they can write dependent functions that work for
any vector size, as opposed to having to use multiple functions for
different sizes.


\subsection{Performance\label{sub:Performance}}

There are ways to avoid the generator step now, however they do not
have acceptable performance or memory layout properties.

The simplest way is to define \texttt{Vector} as dynamicly sized (like arrays).
This simple definition loses all static type safety but does mean functions would only
have to be written once, saving us the effort of creating and maintaining
the generator but a cost. A dynamically sized \texttt{Vector} is also no longer purely a value type
as it will have a reference field in it, this changes the semantics
from the current vector types (due to no user defined copy constructors
or assignment operators in the CLI) and also makes it more expensive
as they are now tracked by the garbage collector. It also makes inter
operating with native APIs such as OpenGL harder, as the Vectors will
have to be marshaled to correctly copy the elements of their internal
array to the native API, the current vector types can just be pinned
and pointer copied due to their flat data layout.

Another way is to define an interface \texttt{Vector} that defines
the indexing operator and length property, and then write functions
using this interface. We still need to create concrete types
for each vector size which requires the generator, saving us little work
and giving dubious benefit.

\begin{lstlisting}[keywordstyle={\color{blue}},language=sharpc]
interface Vector
{
	int Length { get; }
	float this[int index] { get; }
}
public static float Dot<T>(T a, T b) where T : Vector
{
	float dot = 0;
	for(int i=0; i<a.Length; ++i)
	{
		dot += a[i]*b[i];
	}
	return dot;
}
\end{lstlisting}


The issue with this interface approach (and this give some suggestion
as to how we would want to implement dependent types) is that the
loop cannot be unrolled. For high performance code these small
vector types are supposed to be used for it is an unacceptable trade
off, especially as we still have to maintain the generator anyway.
With dependent types we could have better performance characteristics
and with the preferred flat data layout.

While these vector types are the main motivator for value dependence
there are more uses for value dependent types, we explore these in
the background section.

\section[The Common Language Infrastructure]{The CLI}

The CLI\footnote{Common Language Infrastructure} is a specification for a
virtual execution environment, that is implemented by Microsoft's
CLR\footnote{Common Language Runtime} (often confused with the .NET branding)
and the open source Mono project. It is targeted by VB, C\#, F\#, IronPython
and other languages.  It retains a high level of type information, more so than
the JVM \footnote{Java Virtual Machine} (which for example has no concept of
generic types despite Java supporting them\cite{jvm-erasure}).

As a C\# and F\# programmer the CLI is a more attractive specification
to work with. The ability to retain high level type information allows
easy interoperability between separate CLI modules, even with modules
compiled using different languages.

However this interoperability starts to fall apart when languages
add typing extensions that aren't supported by the CLI. Units of measure
in F\#, for example, are erased at compile time; therefore other modules
which consume an F\# module where units of measure were used cannot
see, and be type-checked according to, the units. This loss of typing
information is not ideal, as it reduces interoperability, and so prompts
us to consider adding value dependence as a CLI feature and not just
an extension to a current CLI language such as C\# or F\#. If units
could be written in terms of dependent types then we can \emph{fix}
them, else at least our extension will not suffer the same problem
of interoperability. 

Moreover any new features added to the CLI should be backwards compatible
and efficient, we need to keep in mind the size of the new types and
their instances, the size of the byte code and the speed to process
it and the speed and size of the JITed code.


\section{Contributions}

This project has investigated the background of value dependent types
and looked into how they could be applied to improve the CLI. We show 
how type equality constraints can be added to the CLI to allow the 
expression of Generalized Algebraic Data Types. We also discuss how
values as type parameters could be added, and the issues surrounding them.

The report is split into 3 parts.
\begin{enumerate}
\item A background investigation of value dependent typing and the CLI..
This makes up the first part of this report. We give an overview of a number of 
languages, what sort of value dependence they support, what problem they solve and 
any disadvantages.
\item A specification to add type equality constraints to the CLI.
We describe what changes need to be made to ECMA-335 to support the expression
of type equality constraints at the CLI level and how this extends verifiable code.
We also give a high level description of what would need to be changed in the Mono project
to support such a specification change. The Mono project including all its supporting tools 
and libraries comes to over 5,000,000 lines of code. While Mono is well organized trying to 
learn a project of this size with no prior experience was optimistic. While an implementation
of this system was a goal of this project, time constraints, the complexity of Mono and our 
inexperience with the system has resulted in us achiving this. We have however spent a lot of 
time reading and attempting to understand the Mono source code and feel that recording what needs 
to be changed at a high level is useful information.
\item A test plan for testing extensions to the CLI.
Given that any changes to the CLI ought to be well tested we devised a test plan
detailing how to write tests for a CLI extensions. This test plan allows us to write 
tests in high level languages rather than CLI assembly (or having to emit CLI bytecode 
programmatically). We also show how to test our specification of type equality constraints
without a runtime implementation.
\item A discussion of values as type parameters.
Finally we end with a discussion of how values as type parameters could be added to the CLI.
\end{enumerate}

\chapter{Background}


\section{The CLI}


\subsection{Common intermediate language}

To allow the reader to more easily follow later discussions, we will
first briefly go over the CLI and CIL. For those more familiar with
Java, the CLI can be compared to the JVM and CIL to Java Bytecode.
The CLI runs Common Intermediate Language (CIL) byte code. CIL is
a type rich, stack based assembly language. Below is \emph{Hello World}
in CIL, it shows the loading of a literal string and then calling
a static method. Where possible I have elided instruction offsets
in CIL code.

\begin{lstlisting}[caption={Hello world},language=cil]
.assembly Hello {} 
.assembly extern mscorlib {} 
.method static void Main() 
{     
	.entrypoint     
	.maxstack 1     
	ldstr "Hello, world!"     
	call void [mscorlib]System.Console::WriteLine(string)     
	ret 
}
\end{lstlisting}


CIL supports many features not common for low level assembly code, as well as
basic operations such as add, jump, load, store. Operations such as field
access, method call, object creation, casting etc all have dedicated CIL
instructions. A more concrete example of CIL code is listing \ref{lst:loadppm}. 
Listing \ref{lst:loadppmcsharp} shows LoadPpm, a method from a real C\# application.
This demonstrates the use of locals, calling static and instance methods, method arguments, 
constructors and properties. The corrsponding CLI code is shown in listing \ref{lst:loadppm}.

\begin{lstlisting}[keywordstyle={\color{blue}},language=sharpc,label={lst:loadppmcsharp},caption={LoadPpm C\#}]
private static Image LoadPpm(string name)
{
	var path = System.IO.Path.Combine(Toplevel(), "Memorial", name);
	var ppm = new Ibasa.Media.Visual.Pnm(path);

	return new Ibasa.SharpIL.Image(ppm.Image, 0, 0);
}
\end{lstlisting}
\begin{lstlisting}[caption={LoadPpm CIL},language=cil,label={lst:loadppm}]
.method private hidebysig static 
	class [Ibasa.SharpIL]Ibasa.SharpIL.Image LoadPpm(string name) 
	cil managed 
{
	.maxstack 4
	.locals init (
	[0] string path,
	[1] class [Ibasa.Media]Ibasa.Media.Visual.Pnm ppm)
 
	L_0000: call string Graphics.Program::Toplevel()
	L_0005: ldstr "Memorial"
	L_000a: ldarg.0
	L_000b: call string 
		[mscorlib]System.IO.Path::Combine(string, string, string)
	L_0010: stloc.0
	L_0011: ldloc.0
	L_0012: newobj instance void 
		[Ibasa.Media]Ibasa.Media.Visual.Pnm::.ctor(string)
	L_0017: stloc.1
	L_0018: ldloc.1
	L_0019: callvirt instance class [Ibasa.SharpIL]Ibasa.SharpIL.Resource
		[Ibasa.Media]Ibasa.Media.Visual.Pnm::get_Image()
	L_001e: ldc.i4.0
	L_001f: ldc.i4.0
	L_0020: newobj instance void [Ibasa.SharpIL]Ibasa.SharpIL.Image::.ctor(
		class [Ibasa.SharpIL]Ibasa.SharpIL.Resource, int32, int32)
	L_0025: ret
}
\end{lstlisting}


Therefore, while CIL is targeted by a variety of languages Visual Basic and C\#
match its semantics most closely. We will use C\# code instead
of raw CIL when possible in examples. Some CLI/C\# features are uncommon
in other languages so we we'll briefly go over them. Some of these
features may affect ideas for our extensions, others just help simplify
explanations and code examples.


\subsection{CLI features}

The CLI supports a number of features to provide a comprehensive target for a
variety of languages. The specification defines a type system, a virtual
execution system that provides an environment for executing code, and finally
metadata that provides a structured way to represent all the information the
CLI needs to load types, layout objects, resolve methods, translate CIL,
enforce security and set up runtime context boundaries. 

The CLI specification is made of three main parts, the type system, the
execution system, and metadata. The type system is covered in the next section.

\subsubsection{The type system}

The type system of the CLI is covered in more detail later, in
\ref{sec:cli_type_system} but briefly covered here as well.

Types in the CLI fall into two categories, value types and reference types.
Value types are built-in types such as integers and floats, or user defined
enumerations and structures. Reference types are the built-in types
(\texttt{Object} and \texttt{String}), objects, interfaces, arrays, delegates,
boxed value types, managed and unmanaged pointers.

Value types describe the sequence of bits that make up a value of that type.
Reference types describe the sequence of bits that make up the value but also
describe the location of that value.

\begin{aquote}{ECMA-335 I.8.2.3}
A type fully describes a value if it unambiguously defines the value's
representation and the operations defined on that value.

For a value type, defining the representation entails describing the sequence
of bits that make up the value's representation. For a reference type, defining
the representation entails describing the location and the sequence of bits
that make up the value's representation.
\end{aquote}

While every value has an exact type it is not always possible to determine the
type by looking at the representation of the value. For example the 32 bits
used to represent an integer or a float are indistinguishable. Some types it is
possible to determine the exact type from the value. Objects are self typing as
they explicitly store their type as part of their value.

Some types do not fully describe any values (abstract classes and interfaces).
While every value has a type, it is not possible to infer the type given the
value if the value is of a value type. If the value is a reference type it is
always possible to infer the type.

\subsubsection{The virtual execution system}

The VES\footnote{Virtual Execution System} provides an environment for
executing CIL managed code. Managed code is code that can only be run under the
\emph{management} of a virtual machine (while originally coined to refer to CIL
code only, this terminology has also spread to other \emph{managed} languages
such as Java). Unmanaged code, in contrary, is raw machine code run directly on the
processor. This distinction clearly has some grey area when you consider
operating systems, memory permissions and new CPU based code security methods,
but it's clear enough for our purpose of separating native from managed code. 

It supports some built-in data types, control flow constructs,
an exception handling model and support for the CIL instruction set.

The supported data types are:
\begin{itemize}
\item Signed and unsigned integers of 8, 16, 32 and 64 bits
\item 32 and 64 bit IEEE floating point numbers
\item Native size signed integer
\item Native size unsigned integer, also used as unmanaged pointer
\item Native size floating point number (internal, not externally visible)
\item Native size object reference to managed memory
\item Native size managed pointer
\end{itemize}

All signed integers are implemented with two's-complement arithmetic.

Managed memory supports garbage collection, as such allocated objects do not
not need to be manually deallocated. The locations pointed to by managed
pointers and references can be changed during a garbage collection cycle as
objects are moved in memory. While a managed pointer or reference is still
valid (i.e. reachable), the object it points to will remain in memory.

Exceptions objects are any valid class type (a boxed value type or any class),
but not native integers. When an exception is thrown the runtime searches for the
first \texttt{catch} handler up the stack that covers the throw instruction. If
no \texttt{catch} handler is found then execution is aborted. If a match is found
the stack is wound back to that \texttt{catch} handler, executing all
\texttt{finally} and \texttt{fault} handlers found on the way up.

\subsubsection{Metadata}

Executables and libraries are stored in assemblies. Each assembly is made up of
one or more modules. Each module is made up of type and method declarations.
This information is stored in a structured way and is known as metadata. The
CLI uses metadata to locate and load classes, lay instances out in memory,
resolve method invocations, translate CIL to native code, enforce security and
set up runtime context boundaries.


\subsection{CLI type system}

\label{sec:cli_type_system}

%Why are types important? Important to our project or in general?%


\subsubsection{Value and Reference types}

The CLI (and C\#) differentiates between value types (structs) and
reference types (classes). Value types are allocated inline, either
on the stack or as part of a containing types allocation. Reference
types are allocated on the heap and referred to by a pointer (called
a reference); these are tracked by the garbage collector. To compare
this to C++, \texttt{Foo} is a value type, while \texttt{Foo{*}} is
a reference type, the semantics are similar.


\subsection{Literal and initonly\label{sub:Literal-and-initonly}}

Fields in the CLI can be marked as \texttt{initonly}, and if they are static
fields, \texttt{literal}. Properties do not support either of these modifiers.

A \texttt{static literal} field has no space allocated for it in the metadata,
instead any reference to that field must have the literal value copied into the
use site, as such literal fields must be a primitive type (int, float, string,
etc) In C\# the keyword \texttt{const} is used instead of \texttt{literal}.

An \texttt{initonly} field can only be written to by a constructor method (or
if static by the type initializer method). Other methods can only load from the
field. In C\# the keyword \texttt{readonly} is used instead of
\texttt{initonly}. The property initonly is not transitive, for example the
following C\# mutates a readonly Pair field and is valid code.

\begin{lstlisting}[keywordstyle={\color{blue}},language=sharpc]
class Pair
{
	public int A;
	public int B;

	public Pair(int a, int b)
	{
		A = a;
		B = b;
	}
}

class Program
{
	public readonly Pair MyPair;

	public Program()
	{
		MyPair = new Pair(1, 2);
	}

	static Main()
	{
		var program = new Program();
		//program.MyPair = new Pair(3, 4); not valid
		program.MyPair.A = 3; // valid
	}
}
\end{lstlisting}

This is a very weak concept of immutability, and when adding user
defined types to value dependence could present problems. For example if a 
type was paramaterized on a \texttt{Pair} value, should it be possible to write 
to the \texttt{A} and \texttt{B} fields. If so then clearly our value is the instance
of the \texttt{Pair} not it's data. However what if \texttt{Pair} was a value type (struct)?
A readonly struct can still have it's fields written to, but a struct doesn't have the concept
of an instance it's just data.

\subsection{Properties}

As pointed out above the CIL has instructions for field access but it also has
first class support for properties. Properties support get and set methods
(both optional), which do not have to have the same visibility (it's valid to
have a public get and private set). Properties can also have parameters which
turns them into indexers.  In the CIL code these look similar to method calls
but in C\# they look like field access.

\begin{lstlisting}[keywordstyle={\color{blue}},language=sharpc, escapechar={~}]
class Square {
	public int Length;
	public int Area { get { return Length * Length; } }

	static void Main() {
		Square sq = new Square();
		~\colorbox{yellow}{sq.Length = 4;}~
		Console.WriteLine(sq.Area); // outputs 16
	}
}
\end{lstlisting}


The corresponding CIL follows. Note the method call for \texttt{get\_Area}
on line 16 in \texttt{Main}. While the property getter is just a method,
it is marked up specially in the \texttt{.property} clause so that
other tools can treat it specially. 

\begin{lstlisting}[numbers=left,language=cil,escapechar={~}]
.class private auto ansi beforefieldinit Square     
extends [mscorlib]System.Object 
{ 
	.method private hidebysig static void Main() cil managed     
	{         
		.entrypoint
		.maxstack 2
		.locals init ([0] class Square sq)
   
		newobj instance void Square::.ctor()         
		stloc.0          
		ldloc.0          
		ldc.i4.4          
		stfld int32 Square::Length         
		ldloc.0          
		~\colorbox{yellow}{callvirt instance int32 Square::get\_Area()}~
		call void [mscorlib]System.Console::WriteLine(int32)         
		ret
	}
	
	.property instance int32 Area     
	{
		~\colorbox{yellow}{.get instance int32 Square::get\_Area()}~
	}
	
	.field public int32 Length 

	.method public hidebysig specialname instance int32 get_Area() cil managed
	{
		.maxstack 2

		ldarg.0
		ldfld int32 Square::Length
		ldarg.0 
		ldfld int32 Square::Length
		mul
		ret  
	}
}
\end{lstlisting}

Adding parameters to a property changes it into an indexer. The CIL remains
pretty similar but C\# treats indexers very differently. C\# only allows
indexers called \texttt{this}, \texttt{this} is translated to \texttt{Item}
when compiled to CIL. If a type has an indexer \texttt{Item}, then values of the
type can have an indexer expression appended to them (like accessing an array).

\begin{lstlisting}[keywordstyle={\color{blue}},language=sharpc]
class StringIntMap {
	public int this[string key] { 
		get { ...; } set { ...; } // assuming a sensible implementation
}

void Main() {
	StringIntMap map = new StringIntMap();
	map["test"] = 1;
	Console.WriteLine(map["test"]); // outputs 1
}
\end{lstlisting}

Other languages (such as F\# and C++/CLI) allow named indexers. While they
both treat \texttt{Item} specially in the same way as C\# they allow parameters
on explicitly named properties as well.

\begin{lstlisting}[keywordstyle={\color{blue}},language=haskell]
let row = 1 in matrix.rows[row]
\end{lstlisting}

\subsection{Generics}

The CLI supports parametric polymorphic types via generics types.
Generic types are parametrized on other types (value dependence would
allow types to also be parametrized on values). The MSR White paper
\cite{ext-vox} describes some initial design considerations to do
parametric polymorphism in COM+ (the original name for what became
the CLI and .NET). While the final design and implementation that
shipped with .NET differs slightly from the design presented in \cite{ext-vox},
the paper does give an insight into what we need to be thinking about
while designing value parametrics. One major change between the proposal
and the final implementation is the use of value types as generic arguments.
It was originally thought that allowing value types as generic arguments
was not worth the performance cost of having to re-jit all code that
used them; the final implementation decided that the expressivity and
performance increase from not boxing out weighed this downside. It's
worth taking some time to look at how generics ended up being specified
in ECMA-335\cite{ecma-335} and implemented in Mono (due to copyright
reasons we can't look at Microsoft\textquoteright{}s open source CLR
code). 

Generics are defined in section II.9 of \cite{ecma-335}. A type in
the CLI can have a fixed generic arity (that is generics are not variadic),
the parameters are unnamed and are accessed by index (either !0 or
for type parameters and !!0 for method parameters). Each type parameter
may be constrained by a number of properties, including constraints
on being a value or reference type, having a defined base class or
interface or being default constructable. Type parameters can be value
or reference types; this is a marked difference from the suggestion
in \cite{ext-vox} which suggested that value types should not be
allowed due to having to re-JIT the types code for each value type.

Generics allow the CLI to represent types such as \texttt{List<T>}
while retaining run time information such that the run time type of
\texttt{List<object>} is different to \texttt{List<int>}%
\footnote{In contrast theses types would be equivalent in the JVM.%
}. \texttt{List<int>} is also special in that \texttt{int} is a value
type and yet the run time can use a \texttt{List<int>}without causing
excessive boxing of values.

If we look at the definition of \texttt{List<T>} in Microsoft\textquoteright{}s
distribution of .NET 4.0 we can see how the generic parameter is declared
and used.
\begin{lstlisting}[numbers=left,language=cil]
.class public auto ansi serializable beforefieldinit List`1<T>
extends [mscorlib]System.Object 
implements System.Collections.Generic.IList`1<!0>, 
	System.Collections.Generic.ICollection`1<!0>, 
	System.Collections.Generic.IEnumerable`1<!0>, 
	System.Collections.IList, 
	System.Collections.ICollection, 
	System.Collections.IEnumerable
{ ... }
\end{lstlisting}


The declaration \texttt{.class public auto ansi serializable beforefieldinit
List`1<T>} declares a new class type with one generic parameter \texttt{T},
which has no constraints. The \texttt{implements} clause lists interfaces
implemented by \texttt{List`1<T>}, the first three of these interfaces
are themselves generic. On line 3 the \texttt{System.Collections.Generic.IList`1}
syntax indicates that we mean the generic \texttt{IList} with one
parameter \texttt{`1}, while <!0> refers to the first generic class
parameter \texttt{T}, and passes that as the type argument to \texttt{IList}.

Generic parameters can also be constrained, a run length compressed
list for example would require that the type it stored had an equality
operator. The \texttt{IEquatable<T>} interface defines a method \texttt{bool
Equal(T value)}, so if a type \texttt{T} inherits from \texttt{IEquatable<T>}
then it can be compared equal to other values of its type. Adding
the constraint that the first generic parameters has this property
is shown here. Note the \texttt{(IEquatable`1<!0>)} before the \texttt{T}.

\begin{lstlisting}[language=cil]
.class public auto ansi sealed beforefieldinit
	CompressedList`1<(IEquatable`1<!0>) T> 
extends [mscorlib]System.Object
implements System.Collections.Generic.IEnumerable`1<!0>, 
	System.Collections.IEnumerable
{ ... }
\end{lstlisting}


Sometimes it is necessary to distinguish between instantiated and
non-instantiated generic types. Common terminology for this, as used
in ECMA-335, is close and open generic types. A closed generic type
is one that has no unbound type parameters, conversly an open generic
type is a generic type that has at least one unbound type paramter.

\begin{lstlisting}[caption={Open and closed type in C\# syntax},keywordstyle={\color{blue}},language=sharpc]
Dictionary<TKey, TValue> a; // open
Dictionary<TKey, int> b; // open
Dictionary<string, int> c; // close
\end{lstlisting}

\section{C++ templates\label{sec:Cpp-templates}}

C++ templates allow functions and types to be parametrized by types
or values. Templates are a turning complete language by themselves
making them very general, however most implementations of templates
simply perform substitution at compile time leading to a large amount
of generated code that then has to be reduced by looking for similarities
(in contrast to CLR and Mono generics that duplicate very little code),
and with large use cases substantial slowdowns to compilation time.

Many libraries including the standard library make use of templates,
particularity the parametrization on types. Parametrization on values
is less used but it's similarities to value dependence make it worth
looking at, to this end we will look at a few examples from the standard
library, Boost\cite{boost} and CML\cite{cmldev}. 

The standard C++ library uses value templates in a few places including
\texttt{std::ratio} and the random number generation library. \texttt{std::ratio}
is a compile time rational number added in C++11, it reduces the numerator
and denominator to lowest terms at compile time. The random number
generator uses template values to set generator parameters such as
the constants \texttt{a}, \texttt{c} and \texttt{m} to be used in
\texttt{std::linear\_congruential\_engine}.

\begin{lstlisting}[keywordstyle={\color{blue}},language={C++}]
template<
	class UIntType, 
	UIntType a, 
	UIntType c, 
	UIntType m 
> class linear_congruential_engine;
\end{lstlisting}


The open source Boost\cite{boost} libraries make use of value templates
much more, using them in obvious ways in the Array library, which
is for safer arrays using a new class \texttt{Array<typename T, int
N>}, but also scattered throughout the other libraries. For example
in \texttt{Spirit::Qi}, a parser combinator library, the type \texttt{unit\_parser}
is templated on the type name of the integer type to return but also
on the values of the radix and minimum and maximum digits to parse. 

Finally CML\cite{cmldev} uses value templates to define the sizes
of vectors and matrices, this is similar to our motivating example
in C\#. Vector and matrix are templated on two types \texttt{ElementT}
and \texttt{StorageT}. \texttt{ElementT} is the element type, float,
double, int or another type that supports the same operations. \texttt{StorageT}
is a type that provides access to the elements, either by pointing
to an external data source or storing the data itself. Two of the
built-in storage types (fixed and external) are templated on the value
of how many elements they store/point to. When using these statically
sized storage types, you get extra static type safety that you're not
mixing vector sizes in operations.

\begin{lstlisting}[keywordstyle={\color{blue}},language={C++}]
cml::vector<float, fixed<3>> a(1,0,0);
cml::vector<float, fixed<2>> b = a; // compile error
cml::matrix<float, fixed<2,2>> i(1, 2, 3, 4);
cml::matrix<float, fixed<3,3>> j = i; // compile error
\end{lstlisting}

It's worth noting that although this example looks like the number of paramters 
passed to the constructors match the value passed to fixed, they are 
actually predeclared constructors for a variaty of sizes. Using the wrong 
constructor will either leave some elements uninitialized or not use some of 
the values passed in. Real variadic parameters that matched the dimension of 
the vector/matrix would be better, and with the new features of C++11 might be possible.

%Path dependance?
%Printf example.


\section{F\# units of measure\label{sec:Fsharp-units}}

F\# has the ability to markup number values with units of measure
that allow checking of units at compile time. This extra checking
can prevent mistakes such as that which brought down the Mars Climate
Orbiter in 1999. The Orbiter crashed because of a mismatch between
Imperial and Metric units in force calculation. A very expensive mistake
as the mission cost \$327.6 million\cite{mars}. 

Units of measure are declared as opaque types marked up with the \texttt{Measure}
attribute.
\begin{lstlisting}[keywordstyle={\color{blue}},language=ML]
[<Measure>] type meter
\end{lstlisting}
They can also be declared as equal to other units, for example milliliters
as cubic centimeters.
\begin{lstlisting}[keywordstyle={\color{blue}},language=ML]
[<Measure>] type ml = cm^3
\end{lstlisting}
The normal unit operators, such as multiplication, division and exponentiation,
are usable and can be worked out by the type inference engine. For
example in the following code, type inference correctly identifies
\texttt{distance} as type \texttt{float<meter>}.
\begin{lstlisting}[keywordstyle={\color{blue}},language=ML]
let speed = 55.0<meter/second>
let time = 3.5<second>
let distance = speed * time;

speed    : float<meter/second>
time     : float<second>
distance : float<meter>
\end{lstlisting}
The compiler will normalize units of measure to a standard form, from
the MSDN documentation\cite{fsharp-units-of-measure}
\begin{quotation}
``Unit formulas that mean the same thing can be written in various
equivalent ways. Therefore, the compiler converts unit formulas into
a consistent form, which converts negative exponents to reciprocals,
groups units into a single numerator and a denominator, and alphabetizes
the units in the numerator and denominator.''
\end{quotation}
Units of measure are a common praise of F\# and provided a valuable
case study for us to use in our type system extension. 

F\# units of measure are checked at compile time, implemented as a
sort separate from the standard types. However all units units information
is erased from the run time. Therefore values cast to \texttt{Object}
cannot be recast to a measured type safely at run time, but also these
units cannot be exposed as part of a public interface to be consumed
by other CLI languages such as C\# or VB. 

While they are implemented as a separate sort they behave somewhat
like values of a standard type (with operations for multiplication
and division). A system that allowed them to be values of a Measure
type (rather than a separate sort) while retaining the current features
(including inference) would be impressive and something our system
should strive for.

\begin{lstlisting}[caption={Example Unit type},keywordstyle={\color{blue}},language=sharpc,showstringspaces=false]
public sealed class Unit 
{     
	public static Unit One     
	{         
		get         
		{             
			return new Unit(new List<Tuple<string, int>>());         
		}     
	}
	    
	private readonly List<Tuple<string, int>> Units;
	    
	private Unit(List<Tuple<string, int>> units)     
	{         
		Units = units;     
	}
	    
	public Unit(string unit)     
	{
		Units = new List<Tuple<string, int>>();
		Units.Add(Tuple.Create(unit, 1));
	}
	    
	public static Unit operator *(Unit a, Unit b)
	{
		return new Unit(Product(a.Units, b.Units));
	}
	    
	public static Unit operator /(Unit a, Unit b)
	{
		return new Unit(Product(a.Units, Reciprocal(b.Units)));
	}
	    
	public static Unit operator ^(Unit a, int power)
	{         
		if (power == 0)
			return One;
		return new Unit(Power(a.Units, power));
	}
	    
	private static List<Tuple<string, int>> Normalize(
		List<Tuple<string, int>> units)     
	{         
		var groups = units.GroupBy(tuple => tuple.Item1);         
		var sums = groups.Select(group => 
			Tuple.Create(group.Key, group.Sum(unit => unit.Item2)));         
		var filter = sums.Where(unit => unit.Item2 != 0);        
		var sorted = filter.OrderBy(unit => unit.Item1); 
		return sorted.ToList(); 
	}
	    
	private static List<Tuple<string, int>> Product(
		List<Tuple<string, int>> a, List<Tuple<string, int>> b)     
	{         
		return Normalize(a.Concat(b).ToList());     
	}
	    
	private static List<Tuple<string, int>> Power(
		List<Tuple<string, int>> a, int power)     
	{         
		return a.Select(unit => 
			Tuple.Create(unit.Item1, unit.Item2 * power)).ToList();     
	}
	    
	private static List<Tuple<string, int>> Reciprocal(
		List<Tuple<string, int>> units)     
	{         
		return units.Select(unit => 
			Tuple.Create(unit.Item1, -unit.Item2)).ToList();  
	}
	    
	public override bool Equals(object obj)     
	{         
		if (obj is Unit)         
		{             
			var other = obj as Unit;
			            
			return                  
				Units.Count == other.Units.Count &&                  
				Units.Zip(other.Units, (a, b) => 
					a.Item1 == b.Item1 && a.Item2 == b.Item2).All(b => b);         
		}         
		return false;     
	}
	    
	public override string ToString()     
	{         
		return string.Join(" ", Units.Select(unit => 
			string.Format("{0}^{1}", unit.Item1, unit.Item2)));     
	}
}
    
public static void Example()     
{         
	Unit meters = new Unit("m");         
	Unit seconds = new Unit("s");        
	Unit metersPerSecond = meters / seconds;     
}
\end{lstlisting}



\section{Path dependent types\label{sec:Path-dependent-types}}

Path dependent types like those found in Scala are similar to value
dependent types in that they depend on the value of the object that
created them, but they are not as general. An example of path dependence
in Scala is the following Board and Coordinate example\cite{stack-2693067}.

\begin{lstlisting}[keywordstyle={\color{blue}},language=scala]
case class Board(length: Int, height: Int) 
{   
	case class Coordinate(x: Int, y: Int) 
	{      
		require(0 <= x && x < length && 0 <= y && y < height)    
	}   
	val occupied = scala.collection.mutable.Set[Coordinate]() 
}

val b1 = Board(20, 20) 
val b2 = Board(30, 30)
var b3 = b1
val c1 = b1.Coordinate(15, 15) 
val c2 = b2.Coordinate(25, 25) 
b1.occupied += c1 
b2.occupied += c2
b3.occupied += c1
// Next line doesn't compile 
b1.occupied += c2
\end{lstlisting}


Here the type of \texttt{c1} and \texttt{c2} depend on the values
\texttt{b1} and \texttt{b2}. Not that it is in fact the values not
these specific identifiers that are the dependence, as shown on line
17. Path dependence in the type system does not allow line 19, which
is stricter than just inner classes in Java. 

Path dependence is an extension of the fact that in Scala and Java
inner classes are created via an instance of the outer class and maintain
a reference to their creator. I call the creation via an instance
of the outer class an instance inner types, as opposed to static inner
types that do not require an instance of the outer class. The CLI
does not support path dependent types or instance inner types, the
only difference between inner and outer class in the CLI is viability
(that is an inner class can be made private and thus only be accessed
by the outer class). While it's possible to require a reference to
the outer class as part of the inner class's constructor it is not
a requirement. While instance created inner classes and then path
dependence could be added at the language level this leads to the
risk that Scala ran into where the virtual machine reflection system
no longer resembled the language type system, thus pushing for the
implementation of a whole new reflection system to be built. 

Therefore if we are to investigate adding path dependent
types we also need to add instance inner types to the CLI. Alternatively,
we could try to design value dependence such that the following was
possible.

\begin{lstlisting}[keywordstyle={\color{blue}},language={C++}]
class Board
{
	int length, height;

	public Board(int length, int height)
	{
		this.length = length;
		this.height = height;
	}

	class Coordianate<Board b>
	{
		public Coordinate(int x, int y)
		{
			require(0 <= x && x < b.length && 0 <= y && y < b.height)
		}
	}

	Set<Coordianate<this>> occupied = new Set<Coordinate<this>>;
}
\end{lstlisting}

Allowing the value parameter to be any type is much more general than
path dependence, In this case \texttt{Coordinate} would not even need
to be an inner class of \texttt{Board}. However  this is a very ambitious
addition and if it's even possible is uncertain.

\section{Virtual types\label{sec:Virtual-types}}

Virtual types are also found in Scala, they allow a subclass to override
a type variable in the super class. In the following example the type
\texttt{T} declared in class \texttt{A} is made more specific in the
subclass \texttt{B}.

\begin{lstlisting}[keywordstyle={\color{blue}},language=Java]
class A
{
	type T
	abstract T foo();
}

class B 
{
	override type T = String
	override T foo() { return "string"; }
}
\end{lstlisting}


While virtual types can be useful everything they accomplish can also
be done with generics, albeit with sometime much more syntax. \cite{staticvts}
shows how the same program can be expressed with virtual types or
parametrized types. While one way is often more elegant than the other
you gain little in supporting both. As parametrized types are already
supported by the CLI virtual types are not hugely interesting.


\section{First class types}

Cayenne\cite{cayenne} is a language with support for dependent types
and first class types (i.e. types can be be used like values). As
Cayenne is a functional language inspired by Haskell, it's unlikely
we can lift ideas straight from it to be used in the CLI, however
it provides an example of a very general dependent types system. Two
core features of Cayenne are dependent functions and dependent records.
Dependent functions allow a function return type to depend on the
value of the parameter, as shown in the following example from \cite{cayenne}.

\begin{lstlisting}[keywordstyle={\color{blue}},language=Haskell]
printfType :: String -> #
PrintfType "" = String
PrintfType ('%':'d':cs) = Int    -> PrintfType cs
PrintfType ('%':'s':cs) = Stirng -> PrintfType cs
PrintfType ('%':_:cs)   =           PrintfType cs
PrintfType (_:cs)       =           PrintfType cs

printf :: (fmt::String) -> PrintfType fmt
printf fmt = pr fmt ""

pr :: (fmt::String) -> String -> PrintfType fmt
pr "" res = res
pr ('%':'d':cs) res = \(i::Int) -> pr cs (res ++ show i)
pr ('%':'s':cs) res = \(s::String) -> pr cs (res ++ s)
pr ('%':c:cs) res   = pr cs (res ++ [c])
pr (c:cs) res       = pr css (res ++ [c])
\end{lstlisting}


In this example the type of \texttt{printf} depends on the value of
the parameter fmt. This also shows how types and values are treated
equally in Cayenne. The type \texttt{\#} is the type of all types
(normal notation is \texttt{{*}} but \texttt{\#} was chosen to avoid
clashes with the infix operator \texttt{{*}}). 


\section{Generalized algebraic data types\label{sec:GADTs}}

Generalized algebraic data types (GADTs) are predominately a feature
of functional languages. They are an extension to algebraic data types.
They allow more expression in data type constructors, in particular
they allow pattern matching and more general recursion in a data constructor.
The common example is a type for terms in a small language; with GADTs
it's possible to express constraints on the expression trees that
are not expresable in normal ADTs.

\begin{lstlisting}[caption={GADT},keywordstyle={\color{blue}}]
data Exp t where
	Lit :: Int -> Exp Int
	Plus :: Exp Int -> Exp Int -> Exp Int
	Equals :: Exp Int -> Exp Int -> Exp Bool
	Cond :: Exp Bool -> Exp a -> Exp a -> Exp a
\end{lstlisting}
This data type will only allow correct instantations of \texttt{Exp}
as paramters. The constraint that \texttt{Plus} takes two \texttt{Int}
expressions and returns a new Int expression is expressed. As is the
constraint that \texttt{Cond} must take a \texttt{Bool} expression
and two other expression of the same type returning an expression
of that type. Without GADTs these constraints cannot be expressed,
the following shows the same type as an ADT.

\begin{lstlisting}[caption={ADT},keywordstyle={\color{blue}},language=Haskell]
data Exp 
	= Lit Int
	| Plus Exp Exp 
	| Equals Exp Exp
	| Cond Exp Exp Exp
\end{lstlisting}


The first expression passed to \texttt{Cond} must evaluate to a boolean
result (from \texttt{Equals}), but the type system cannot express
that. The following expression tree is valid with the ADT type, and
invalid with GADTs.

\begin{lstlisting}[language=Haskell]
Cond (Lit 1) (Lit 2) (Equals (Lit 3) (Lit 4))
\end{lstlisting}


There are more examples that can be statically checked with GADTs
such as lists that have their size as part of their type and statically
typed printf functions.

The use of GADTs in object orientated languages is less common than
in functional languages (Haskell has supported GADTs for over 10 years)
but \cite{gadts} shows how GADT programs can be expressed in C\#
with some modifications to the language. The two modifications proposed
by \cite{gadts} are an extension of generic constraints and an extension
of the switch statement. 

The extension to generic constraints would allow equality constraints
on generic types, section 3.1 (Equational constraints for C\#) of
\cite{gadts} describes this extension. This would allow a generic
type to be declared equal to another type, this would be checked statically
at compile time. 

We'll use a different example from the expressions code above, as
the expression type is much larger expressed in C\#. Instead we'll
look at list flatten methods. A list flatten method could check that
the list was a list of lists by the addition of the \texttt{where
T=List<U>} clause.

\begin{lstlisting}[caption={C\# GADT},label={lis:csharp gadt},keywordstyle={\color{blue}},language=sharpc]
public abstract class List<T> {
	public abstract List<T> Append(List<T> list);
	public abstract List<U> Flatten<U>() where T=List<U>;
}

public class Nil<T> : List<T> {
	public override List<U> Flatten<U>() {
		return new Nil<U>;
	}
}

public class Cons<T> : List<T> {	
	T head; List<T> tail;
	public override List<U> Flatten<U>() {
		return this.head.Append(this.tail.Flatten());
	}
}
\end{lstlisting}


Calling \texttt{Flatten} on a \texttt{List<T>} would statically check
that \texttt{T=List<U>} where \texttt{U} is any type. Thus in the
method body of flatten we can assume that the type of \texttt{head}
is \texttt{List<U>} which has an \texttt{Append} method. While \cite{gadts}
suggests this addition of type equality constraints as a C\# extension;
generic constraints are currently encoded at the CLI level and so
we could add this as a CLI extension. Adding type equality constraints
at the CLI level would allow it to be added to C\# and other languages
more easily. 

The second proposal is an extension to the switch statement to allow
switching on types, binding type \texttt{variables} in switch case
clauses and matching multiple expressions.

\begin{lstlisting}[keywordstyle={\color{blue}},language=sharpc]
switch (e1, e2)
	case (Lit x, Lit y):
		return x.value == y.value;
	case (Tuple<A,B> x, Tuple<C,D> y):
		return Eq(x.fst, y.fst) && Eq(x.snd, y.snd);
	default:
		returna false;
}
\end{lstlisting}


While standard switch statements are a language feature (at the CLI
level they are encoded through a sequence of if statements) the authors
point out that support at the CLI level for a match-and-bind primitive
would be useful (see the end of section 3.4 in \cite{gadts}) as their
switch extension is currently difficult to translate to CIL code,
having to rely on run time reflection and generic methods to bind
correctly.

%Add printf example%

%Relation to type classes? Why do we need equality constraints?%


\section{Conclusion}

Having looked at all these type systems we can see that some systems
are more powerfull, while other are equal in power but differ in expresivity.

\textbf{C++ templates} being Turing complete are the most powerful
system we've looked at, but that comes with it's downsides. Efficiently
compiling templates such that the final code is small and fast is
difficult, they also make the language unsound as a template can recurse
forever (although most compilers have hard limits to this).

The next most general are \textbf{value dependence}. Value dependence
allowed types to be constructed based on values. While this is powerful
allowing arbitrary values as parameters is undecidable, as it amounts
to determining whether two different programs produce the same result.
In chapter 30.5 of \cite{tapl} is a warning about dependent types:
\begin{quotation}
Unfortunately, the power of dependent types is a two-edged sword.
Blurring the distinction between checking types and carrying out proofs
of arbitrary theorems does not magically make theorem proving simple
- on the contrary, it makes type checking computationally intractable!
Mathematicians working with mechanical proof assistants do not just
type in a theorem, press a button, and sit back to wait for a Yes
or No: they spend significant effort writing proof scripts and tactics
to guide the tool in constructing and verifying a proof. If we carry
the idea of correctness by construction to its limits, programmers
should expect to expend similar amounts of effort annotating programs
with hints and explanations to guide the type checker. For certain
critical programming tasks, this degree of effort may be justified,
but for day-to-day programming it is almost certainly too costly.
\end{quotation}
The CLI as a mainstream day-to-day infrastructure would certainly
not benefit from an extension that required significant expenditure
of programmer time. As such, we do not actually want to make our system
too powerful, we want to find a balance between opening up opportunities
for optimization and expressivity and the cost of annotation and understanding.

\textbf{GADTs} come in next. Section \ref{sec:GADTs} explored how
GADTs can be expressed in C\# with some modifications. As generic
constraints are already stored in the metadata, taking these ideas
and expanding them to cover type equality should be simple. 

We've seen how \textbf{virtual types} are equivalent to generics.
As the CLI already supports generics further investigation of virtual
types seems unnecessary. Finally, path dependence is a simpler case
of value dependence as are F\# units of measure. 

While full value dependence may be too much, GADTs aren't enough
leading us to think about an extension somewhere in between the two.

\[
\begin{array}{c}
Templates\\
Value\; dependence\\
Our\; extension?\\
GADTs\sim Generics+Type\; equality\; constraints\\
Virtual\; types\sim Generics
\end{array}
\]

The following table shows some of the differances and similarites
between these systems. The Type safe printf column indicates if the
system can express a printf like function in a type safe way. The
type sized lists column indicates if the system can express lists
that carry their size as part of their type.

\begin{center}
\begin{longtable}{|>{\centering}p{0.12\linewidth}|>{\centering}p{0.12\linewidth}|>{\centering}p{0.12\linewidth}|>{\centering}p{0.12\linewidth}|>{\centering}p{0.12\linewidth}|>{\centering}p{0.12\linewidth}|>{\centering}p{0.12\linewidth}|}
\cline{2-7} 
\multicolumn{1}{>{\centering}p{0.12\linewidth}|}{} & Turing complete & Type safe printf & Type sized lists & Decidable & Units of measure & Path dependence\tabularnewline
\hline 
Templates & Yes & Yes & Yes & No & Yes & ?%
\footnote{Being Turing complete it feels this should be yes, however we can't
find any material to suggest either way%
}\tabularnewline
\hline 
Value dependence & No%
\footnote{Agda and Coq aren't Turing complete.%
} & Yes & Yes & No & Yes & Yes\tabularnewline
\hline 
GADTs & No & Yes & Yes & Yes & No%
\footnote{Yes if we allow new Kinds\cite{gadtsext}.%
} & No\tabularnewline
\hline 
Generics & No & No & No & Yes & No & No\tabularnewline
\hline 
\end{longtable}
\par\end{center}

The CLI currently inhabits the lower levels of our list above, only supporting Generics. 
Our first point of work is to add type equality constraints, moving the CLI up in what it can express
and given an equivalent system to GADTs.

Following the specification of type equality constraints, we will discuss values as type parameters. These
allow more programs to be checked by the type system and also open up optimization benefits but are a more
complex extension than equality constraints.

\chapter{Type equality constraints}

\section{Generalized algebraic data types}

The first extension will be to add type equality constraints to the CLI. 
Type equality constraints for C\# are described in \cite{gadts}, we will 
be using these ideas but for the CLI not C\#. The basic idea is to extend 
generic constraints to allow equality constraints on generic types, section 
3.1 (Equational constraints for C\#) of \cite{gadts} describes this extension. 
This would allow a generic type to be declared equal to another type, this 
would be checked statically at compile time. For example a list flatten method
could check that the list was a list of lists by the addition of the 
\texttt{where T=List<U>} clause.

\begin{lstlisting}[keywordstyle={\color{blue}},language=sharpc,tabsize=4]
public abstract class List<T> {
	...
	public abstract List<T> Append(List<T> list);
	public abstract List<U> Flatten<U>() where T=List<U>;
}

public class Nil<T> : List<T> {
	public override List<U> Flatten<U>() {
		return new Nil<U>;
	}
}

public class Cons<T> : List<T> {	
	T head; List<T> tail;
	public override List<U> Flatten<U>() {
		return this.head.Append(this.tail.Flatten());
	}
}
\end{lstlisting}

Calling \texttt{Flatten} on a \texttt{List<T>} would statically check
that \texttt{T=List<U>} where \texttt{U} is any type. Thus in the
method body of flatten we can assume that the type of \texttt{head}
is \texttt{List<U>} which has an \texttt{Append} method. While the
paper suggests this as a C\# extension generic constraints are currently
encoded at the CLI level and so we could add this as a CLI extension,
thus allowing this to be added to C\# and other languages easily. 

\subsection{Example}

The following shows a minimal list example, in both C\# and CIL.


\begin{lstlisting}[caption={Type equality constraints in extended C\#\protect \\
Extension of listing \ref{lis:csharp gadt}},escapechar={~},keywordstyle={\color{blue}},language=sharpc]
public abstract class List<T>
{
	public abstract List<T> Append(List<T> list);
	public abstract List<U> Flatten<U>() ~\colorbox{yellow}{where T=List<U>;}~
}

public class Nil<T> : List<T>
{
	public override List<T> Append(List<T> list)
	{
		return list;
	}
	public abstract List<U> Flatten<U>() // type constraints are inherited
	{
		return new Nil<U>();
	}
}

public class Cons<T> : List<T>
{
	T Head;
	List<T> Tail;

	public Cons(T head, List<T> tail) 
	{
		Head = head;
		Tail = tail;
	}

	public override List<T> Append(List<T> list)
	{
		return new Cons<T>(Head, Tail.Append(list));
	}

	public override List<U> Flatten<U>() // type constraints are inherited
	{
		return Head.Append(Tail.Flatten<U>()); // invalid in standard C#
	}
}
\end{lstlisting}

\begin{lstlisting}[caption={Corresponding CIL},escapechar={~},language=cil]
.class public abstract auto ansi beforefieldinit List<T>
extends [mscorlib]System.Object 
{
	.method family hidebysig specialname rtspecialname instance void .ctor() 
		cil managed
	{
		.maxstack 8
		ldarg.0
		call instance void [mscorlib]System.Object::.ctor()
		ret
	}

	.method public hidebysig newslot abstract virtual instance class 
		Test.List`1<!T> Append(class Test.List`1<!T> list) cil managed     
	{     }

	.method public hidebysig newslot abstract virtual instance class 
		Test.List`1<!!U> Flatten<~\colorbox{yellow}{= T List<!!0>}~ U>() cil managed
	{     }
}

.class public auto ansi beforefieldinit Nil<T>     
	extends Test.List`1<!T> 
{     
	.method public hidebysig specialname rtspecialname instance void .ctor() 
		cil managed     
	{
		.maxstack 8
		ldarg.0
		call instance void Test.List`1<!T>::.ctor()
		ret
	}

	.method public hidebysig virtual instance class 
		Test.List`1<!T> Append(class Test.List`1<!T> list) cil managed     
	{
		.maxstack 1
		ldarg.1
		ret
	}

	.method public hidebysig virtual instance class 
		Test.List`1<!!U> Flatten<~\colorbox{yellow}{= t list<!!0>}~ U>() cil managed
	{
		.maxstack 1
		newobj instance void Test.Nil`1<!!U>::.ctor()
		ret
	}
}

.class public auto ansi beforefieldinit Cons<T>     
extends Test.List`1<!T> 
{
	.method public hidebysig specialname rtspecialname instance void 
		.ctor(!T head, class Test.List`1<!T> tail) cil managed     
	{         
		.maxstack 2        
		ldarg.0
		call instance void Test.List`1<!T>::.ctor()
		ldarg.0
		ldarg.1
		stfld !0 Test.Cons`1<!T>::Head
		ldarg.0
		ldarg.2
		stfld class Test.List`1<!0> Test.Cons`1<!T>::Tail
		ret
	}
	
	.method public hidebysig virtual instance class 
		Test.List`1<!T> Append(class Test.List`1<!T> list) cil managed     
	{         
		.maxstack 3
		ldarg.0
		ldfld !0 Test.Cons`1<!T>::Head
		ldarg.0
		ldfld class Test.List`1<!0> Test.Cons`1<!T>::Tail
		ldarg.1
		callvirt instance class Test.List`1<!0> 
			Test.List`1<!T>::Append(class Test.List`1<!0>)
		newobj instance void Test.Cons`1<!T>::.ctor(!0, class Test.List`1<!0>)
		ret
	}

    .method public hidebysig virtual instance class
		Test.List`1<!!U> Flatten<~\colorbox{yellow}{= T List<!!0>}~ U>() cil managed     
	{         
		.maxstack 2
		nop
		ldarg.0
		ldfld !0 Test.Cons`1<!T>::Head
		ldarg.0
		ldfld class Test.List`1<!0> Test.Cons`1<!T>::Tail
		callvirt instance class Test.List`1<!!0> 
			Test.List`1<!T>::Flatten<!!U>()

		// the following callvirt would not verify in the standard CLI

		callvirt instance class Test.List`1<!0> 
			Test.List`1<!!U>::Append(class Test.List`1<!0>) 
		ret
	}

    .field private !T Head
    .field private class Test.List`1<!T> Tail
}
\end{lstlisting}

The syntax here is purely to demonstrate the intuition of the feature.
Exact syntax will be expanded on as we explore how this can be added
to the CLI specification.

\section{Specification changes}

\subsection{Generic constraints}

Section II.9.11 (Constraints on generic parameters)\cite{ecma-335} specifies
generic constraints. A type parameter that has been constrained must
be instantiated with an argument that is assignable to each of declared
constraints, and that satisfies all special constraints.

The special constraints are, \texttt{+}, \texttt{-}, \texttt{class},
\texttt{valuetype} and \texttt{.ctor}. \texttt{class} constrains the
argument to be a reference type. \texttt{valuetype} constrains the
argument to be a value type, except for any instance of\texttt{ System.Nullable<T>}.
\texttt{.ctor} constrains the argument to a type that has a public
default constructor (implicitly this means all value types as value
types always have a public default constructor). Finally \texttt{+}
and \texttt{-} are used to denote the parameter is covariant or contravariant
respectively.

While it might seem that this is a good place to add our extension
type equality constraints are a constraint on the entire parameter
list, not on each individual parameter. There's also the potential
to add an equality constraint to a non-generic method.

\begin{lstlisting}[keywordstyle={\color{blue}},language=sharpc]
class Foo<T>
{
	public void Bar(List<int> list) where T = int
	{
		...
	}
	
	public void Baz(List<string> list) where T = string
	{
		...
	}
}
\end{lstlisting}

In this example \texttt{Bar} can only be called if \texttt{Foo<T>}
was initialized with \texttt{int}, and \texttt{Baz} only if it was
initialized with \texttt{string}. A similar thing can be done with
non-generic inner types. So we need to look to add this new syntax
somewhere separate to the generic parameter list. Preferably it would
have similar syntax for both methods and types (as generic parameters
look the same on a type declaration or method declaration). A type
declaration currently follows the pattern ``\texttt{.class ClassAttr{*}
Id {[}'<' GenPars '>'{]} {[}extends TypeSpec {[} implements TypeSpec{]}
{[}',' TypeSpec{]}{*}{]}}'', while method declarations follow the
pattern ``\texttt{.method MethAttr{*} {[}CallConv{]} Type {[}marshal
'(' {[}NativeType{]} ')'{]} MethodName {[}'<' GenPars '>'{]} '(' Parameters
')' ImplAttr{*}}''.

Adding a new clause ``\texttt{where {[}Type '=' Type{[}',' Type '=' Type{]}{*}{]}}
to method declarations after the parameter list gives us a list of Types that must 
be equal to other types. It's not strictly necessary to have this clause on type 
declarations as for top level types it makes very little sense and for inner types 
it can be emulated be adding the clause to each method.

\subsection{Assignment compatibility}

Assignment compatibility is defined in section I.8.7 of \cite{ecma-335}, further to this 
verification assignment compatibility is defined in III.1.8.1.2.3. Verification assignment 
compatibility is mostly defined in terms of general assignment compatibility from I.8.7.3.

Verification assignment compatibility is used be the verifier for determining if method calls, field 
references and loads and stores are valid for a given type and signature. If verification assignment compatibility 
is extended to understand type equality constraints then operations that were unverifiable but type correct can now
be checked as verifiable as well.

Adding another rule to \emph{verifier-assignable-to} to use the rules for equality constraints is all that is needed
to enhance this part of the system.

\textbullet{} T is \emph{equal-to} U.

\subsubsection{\emph{equal-to}}

\emph{equal-to} is used to determine if two type names refer to the same actual type.
It uses the both the global typing environment $\Gamma$ and the
equality constraints on the current method $\epsilon$ which defines equal types.

\begin{prooftree}
\AxiomC{$T=U \in \epsilon$}
\LeftLabel{eq-hyp}
\UnaryInfC{$\Gamma,\epsilon \vdash T=U$}
\end{prooftree}

\begin{prooftree}
\AxiomC{$T=U \in \epsilon$}
\AxiomC{$\Gamma \vdash C<T> \,\mathit{ok}$}
\LeftLabel{eq-con}
\BinaryInfC{$\Gamma,\epsilon \vdash C<T>=C<U>$}
\end{prooftree}

\begin{prooftree}
\AxiomC{$\Gamma,\epsilon \vdash C<T>=C<U>$}
\LeftLabel{eq-decon}
\UnaryInfC{$\Gamma,\epsilon \vdash T=U$}
\end{prooftree}

\begin{prooftree}
\AxiomC{$\Gamma \vdash T \,\mathit{ok}$}
\LeftLabel{eq-refl}
\UnaryInfC{$\Gamma,\epsilon \vdash T=T$}
\end{prooftree}

\begin{prooftree}
\AxiomC{$U=T \in \epsilon$}
\LeftLabel{eq-sym}
\UnaryInfC{$\Gamma,\epsilon \vdash T=U$}
\end{prooftree}

\begin{prooftree}
\AxiomC{$\Gamma,\epsilon \vdash T=U$}
\AxiomC{$\Gamma,\epsilon \vdash U=V$}
\LeftLabel{eq-sym}
\BinaryInfC{$\Gamma,\epsilon \vdash T=V$}
\end{prooftree}

\subsection{Method calls}

Any method calls must be checked that any equality constraints on the method being called 
can be satisfied by the current context. This is done by checking that for each constraint listed on the 
method ($T=U$) T and U are \emph{equal-to} in the current context after applying substitution for generic parameters.

\begin{prooftree}
\AxiomC{$\Gamma \vdash m : <\bar{U}> \,\mathit{where}\, \epsilon'$}
\AxiomC{$\forall e \in (\epsilon'[\bar{T}/\bar{U}]).\Gamma, \epsilon \vdash e$}
\LeftLabel{call}
\BinaryInfC{$\Gamma,\epsilon \vdash m<\bar{T}>$}
\end{prooftree}

This is an addition to the verifiability rules for the call, calli and callvirt instructions.

\subsection{Metadata tables}

Metadata tables are specified in section 22 (Metadata logical format:
tables). We're mostly interested in section 22.21 (GenericParamConstraint),
which explains how generic constraints are stored in the assembly. 
\begin{quotation}
The GenericParamConstraint table has the following columns:

\textbullet{} Owner (an index into the GenericParam table, specifying
to which generic parameter this row refers) 

\textbullet{} Constraint (an index into the TypeDef, TypeRef, or TypeSpec
tables, specifying from which class this generic parameter is constrained
to derive; or which interface this generic parameter is constrained
to implement; more precisely, a TypeDefOrRef (�24.2.6) coded index) 

The GenericParamConstraint table records the constraints for each
generic parameter. Each generic parameter can be constrained to derive
from zero or one class. Each generic parameter can be constrained
to implement zero or more interfaces. 

Conceptually, each row in the GenericParamConstraint table is owned
by a row in the GenericParam table. 

All rows in the GenericParamConstraint table for a given Owner shall
refer to distinct constraints.
\end{quotation}
We need a similar table for our equality constraints. It will need
an owner (a type or method) and then two types that are constrained
to be equal within the owner. Another column that isn't necessary
but could be useful is a flags field describing the relationship between
the two types, currently this would always be set to equal, however
the system could be extended at a later date to allow less than and
greater than relationships as well.

So the TypeRelationshipConstraint table has the following columns:
\begin{itemize}
\item Owner (an index into the MethodDef table, specifying the
Method to which this constraint applies)
\item Flags (a 1-byte flag bitmask currently always set to 0. To be used
for extensions)
\item ConstraintA (an index into the TypeDef, TypeRef, or TypeSpec tables,
specifying the first type; more precisely, a TypeDefOrRef (�24.2.6)
coded index)
\item ConstraintB (an index into the TypeDef, TypeRef, or TypeSpec tables,
specifying the second type; more precisely, a TypeDefOrRef (�24.2.6)
coded index)
\end{itemize}

\section{Implementation changes}

\subsection{Runtime}

The runtime needs to be changed to load and understand equality constraint
metadata, these changes will be focused in mono/metadata/metadata.h/c and
mono/metadata/reflection.h/c.  The TypeRelationshipConstraint table is very
similar to the GenericParamConstraint table, the current code for
GenericParamConstraint can guide the new code for TypeRelationshipConstraint.

Verification is done in verify.c. Additions need to be made to verify that type
relationship constraints are met when calling methods and to extend assignment
compatibility to use those type relationship constraints.  This requires
additions to \texttt{mono\_method\_verify} to check constraints on
\texttt{call}, \texttt{calli} and \texttt{callvirt} instructions, this can
probably be done in \texttt{do\_invoke\_method}.  It also requires additions to
\texttt{verify\_stack\_type\_compatibility\_full} to check the new assignment
compatibility rules.

\subsection{corlib}

The core library needs to be changed to support type equality. Specifically the
reflection library defined in mcs/class/corlib/System.Reflection\[.Emit\]. This
API is predominately implement via internal calls that are then defined in
mono/metadata/icall-def.h.

\subsection{Cecil and IKVM}

Mono.Cecil and IKVM are well known libraries for working with CIL code. They
are generally considered better than the core library System.Reflection and
System.Reflection.Emit. Any extensions of the system would be well served by
extending these libraries to understand the extension as well.

\section{Testing}

As part of any implementation a number of test cases should be developed to
show that the system works. These tests need to show that there is no CLI code
that does not run in the new system (backwards compatibility) and that the new
system is correct.

Normally these tests would be written in CIL assembly, however writing tests in
CIL assembly would be time consuming and error prone. Instead test writing
should be assisted by an assembly rewriter. The rewriter will look for calls to
special methods and attributes in a CLI assembly and rewrite the bytecode and
metadata to use features in the extended system.  Assembly rewriting will be
done using the CLI metadata reflection, writing and rewriting via
\texttt{System.Reflection} or another metadata editor such as
\textt{Mono.Cecil}.  Any implementation will extend these objects to support
the new metadata constructs.  With this we can mark up standard C\# or F\# code
with special methods and attributes and rewrite the resulting assemblies to
include the new metadata.

This system of testing has a number of advantages. Firstly it allows us to
write our tests in a high level language such as C\# instead of low level CIL.
The main benefit of writing in a high level language should be clear, it's much
easier. However we also get another less obvious benefit, that we are able to
test both imperative (C\#) and functional (F\#) code. This should mean coverage
over most aspects of the CLI.

Secondly we can run the tests on the standard CLI, this gives us something to
test against. Running in the standard CLI is not equivalent to running the
rewritten program in the new CLI (which we will call CLI+), but does allow some
deductions to be made.

For type equality there are two methods we need to define. \texttt{U Cast<T,
U>(T obj)} that checks \texttt{T} and \texttt{U} are equal at runtime and casts
\texttt{obj} to type \texttt{U} if they are; and \texttt{void EqualTypes<T,
U>()} that will check that T and U are equal types at runtime. Both these
methods will throw an exception \texttt{TypeEqualityException} if \texttt{T}
and \texttt{U} are not equal.

The rewriter will search an assembly for uses of \texttt{Cast<T, U>} and
\texttt{EqualTypes<T,U>}. Any use of \texttt{EqualTypes<T,U>} will be removed
and the method metadata rewritten to include the new type equality tags. Any
use of \texttt{Cast<T, U>} will also rewrite the metadata to include the new
type equality tags and will also remove the call to \texttt{Cast}.

\begin{lstlisting}[caption={Cast},keywordstyle={\color{blue}},language=sharpc]
public static class TypeEquality
{
	public static U Cast<T, U>(T obj)
	{
		if(typeof(T) == typeof(U))
		{
			return (U)(Object)obj;
		}
		else
		{
			throw new TypeEqualityException();
		}
	}
	public static void EqualTypes<T, U>()
	{
		if(typeof(T) != typeof(U))
		{
			throw new TypeEqualityException();
		}
	}
}
\end{lstlisting}

A program $P$ is the source code in a CLI language, such as C\#. It may or may
not make use of \texttt{TypeEquality.Cast} and
\texttt{TypeEquality.EqualTypes}. It can be compiled by a compiler $C$ to give
a CIL assembly, this assembly can be run on a runtime to give a value. The two
runtimes are $CLI$ and $CLI+$, values are either some value $v$ representing
the overall act of computation done by the program or an exception thrown by
the program, the only exceptions we care about are $TypeEqualityException$, or
$VerificationException$.  Finally a CIL assembly can be rewritten by the
rewriter $R$ to produce a CIL+ assembly.

If $R$ and CLI+ are correct then the following statements should hold:
\begin{lemma}
$C(P) \underset{CLI}{\rightarrow} TypeEqualityException \implies R(C(P))
\underset{CLI+}{\rightarrow} VerificationException$
\end{lemma}
\begin{lemma}
$R(C(P)) \underset{CLI+}{\not\rightarrow} VerificationException \implies
C(P) \underset{CLI}{\not\rightarrow} TypeEqualityException$
\end{lemma}
\begin{lemma}
$C(P) \underset{CLI}{\rightarrow} v \wedge R(C(P))
\underset{CLI+}{\not\rightarrow} VerificationException \implies R(C(P))
\underset{CLI+}{\rightarrow} v$
\end{lemma}
\begin{lemma}
$C(P) \underset{CLI}{\rightarrow} v \iff C(P) \underset{CLI+}{\rightarrow} v$
\end{lemma}

Lemma 1 states that if a program throws a \texttt{TypeEqualityException} in
the standard CLI then it will throw a \texttt{VerificationException} in the new
CLI. If CLI+ doesn't throw a \texttt{VerificationException} then we know
something is wrong with either the rewriter or the new runtime. For the sake of
these tests we will assume that the rewriter is correct. This property does not
hold in reverse ($R(C(P)) \underset{CLI+}{\rightarrow} VerificationException
\implies C(P) \underset{CLI}{\rightarrow} TypeEqualityException$) as the call
to \texttt{Cast<T, U>} or \texttt{EqualTypes<T, U>} might not be hit be every
control flow path.

Lemma 2 states that if the rewritten program runs in CLI+ and does not throw a
\texttt{VerificationException} then running the code in the CLI will not throw
a \texttt{TypeEqualityException}. If a \texttt{TypeEqualityException} is thrown
then something is wrong with the new runtime.

Lemma 3 states that if the program computes a value $v$ in the standard CLI
and verifies correctly in the new CLI then the new CLI should compute the same
value $v$.

Lemma 4 states that if the same CLI assembly (no rewriting) is run on both
runtimes they should compute the same value. While this is similar to lemma 3
it's difference is the lack of any assembly rewriting. 

All this together means that we can do some verification of our new system
against the old system. If a program ran correctly in the old system it should
also run correctly in the new system (lemma 4). If a program has correct equality type
constraints then it should run in the old system (and by the lemma 4 also
run in the new system) and when rewritten it should run in the new system
(lemma 3). If a program throws a \texttt{TypeEqualityException} then it should
throw a \texttt{VerificationException} when rewritten and run in the new
system.

\section{Testing without a full implementation}

While the above test plan is appropriate for checking an implementation is correct
we do not have a real implementation. However we still wish to show that the system
could work and is statically checkable. To do this we will be using a similar system to 
above. We can continue to write tests as described above, however instead of using 
an assembly rewriter to transform them into an extended CIL format we simply perform
the checks that the extended verifier would perform. Thus for every method call we 
check to see if it calls \texttt{EqualTypes} and if so check that we can satisfy 
those constraints, and for every call to \texttt{Cast} we check that the two types
are indeed equal either directly or via equality constraints.


\chapter{Values as type parameters}

Our second extension is the addition of values as type parameters to the CLI.
Drawing on what we've learnt in our background research, our knowledge of the 
CLI and the design goals laid out by our motivating example, we discuss some of
our ideas and issues with value dependent types.

These ideas will be often discussed in terms of high level design, but 
must always take into account that they are really targeted at the CLI level.

\subsection{Concept of equality}

To be able to say if a type \texttt{T} that depends on a value of
\texttt{U} (that is \texttt{T<U a>}) is equal to \texttt{T<U b>} requires
us to say what it means for \texttt{a} and \texttt{b} to be equal.
All CLI objects have a method \texttt{bool Equal(object obj)} which
we could use, however this is clearly unsound as an implementation
of \texttt{Equal} could return nonsense, or never return at all.

Instead, we propose using structural equality. That is,
two values of any reference type are equal if and only if they are
the same reference, and two values of a value type are equal if and
only if all their fields are equal in this manner as well. 

In practice, this amounts to checking that the values have the same bytes in memory.
At runtime this is an easy check to make and dynamic type checks and reflection would have
no issues with it. Doing this at compile time requires that each instantiation of a value 
parameter is constant. For value types this is simple to achieve while still being easily usable.
Two separate instantiations of a type \texttt{T<int i>} with the same \texttt{int} value will both end 
up with the same byte values for the \texttt{int} value (although how and where this value is stored 
still needs to be discussed). For custom value types some extra work is required to make sure they are
immutable, but instantiation remains simple. 

However, given some reference type \texttt{U} with a constructor \texttt{U(int i)} two separate instantiations of 
a type \texttt{T<U u>} done via two \texttt{new} expressions will result in two distinct references. 
Listing \ref{lst:refissue} demonstrates this issue. The parameter passed to the inner \texttt{T} constructor of the second 
line is a different instance of \texttt{U} to that present in the first line.
Given that all these values must be constant at compile time its not clear how this could work. We can't 
reference the value from another type specification as that itself requires the reference.

\begin{lstlisting}[label={lst:refissue},caption={An issue with reference values},keywordstyle={\color{blue}},language=sharpc]
var myT = new T<new U(1)>();
var myOtherT = new T<T<new U(1)>.u>();
}
\end{lstlisting}

We could extend the definition of \texttt{literal}(\texttt{const} in C\#) to allow constants of custom types. This would allow us to 
initialize and store the reference once and then use it in type parameters.

\begin{lstlisting}[caption={Literal references},keywordstyle={\color{blue}},language=sharpc]
const U U1 = new U(1);

public T<U1> SomeMethod(T<U1> t)
{
	return t;
}
\end{lstlisting}

While allowing any type as a constant seems ideal, due to the way \texttt{literal} works this would cause significant 
difficulty for reference types, but it would work for value types and would solve the storage issue mentioned earlier, 
so still a wanted extension.
The reason for the difficulty is that uses of \texttt{literal} fields replace the use site with the value at compile time. 
The value for a reference is a managed reference pointer, deciding where that pointer should point is impossible at compile time

\subsection{Immutability}

Type preservation means that an expression's type should not change
under evaluation, therefore value type parameters should be immutable.
As shown in subsection \ref{sub:Literal-and-initonly}, the CLI does
not have strong support for immutability. As such, our initial work
will concentrate on using the primitive types as their immutability
can be easily guaranteed.

As noted above, the values used as parameters should be immutable, allowing these values to change at runtime would be unsound 
as the system has already made judgements based on the static value. For primitive types immutability is easily achieved by making 
it impossible to write to or take the address of the parameter. This also covers us for custom value types, as although instance 
methods on value types take a managed pointer to the value (which could then be used to write to the fields) there is no way to
get this reference in the first place. The value must be copied to a local variable and then instance methods can be called on that.

For references the value we care about is the reference itself, the object pointed to does not need to be immutable.

\subsection{Operations}

Once we have the ability to mark up types with values, we will want
to use the value for operations. Either for changing the value before
passing it on as another type parameter or for using at runtime. 
For example to be able to concatenate two fixed arrays together as shown in listing \ref{lst:arrayconcat}.

\begin{lstlisting}[label={lst:arrayconcat},caption={Concatenate two arrays},keywordstyle={\color{blue}},language=sharpc]
public class Array<T, int length> { 
	...
	
	public Array<T, length + n> Concatenate<int n>(Array<T, n> other) { ... }
}
\end{lstlisting}

Support for using the value in normal methods at runtime seems trivial, just
expose it similar to a static readonly field. However, supporting the
ability to do operations on value parameters before passing them to
another type constructor is more challenging. Firstly, it will require
some effort to fit into the CIL bytecode, currently opcodes are only
allowed in method bodies, we would have to either point to a method
to calculate the operations on value parameters or find some other
way to fit opcodes at the declaration level. Secondly, we have the
issue of soundness as user defined operations can do anything.

Another soloution to supporting operations would be to define a mini expression langaguge. This might 
support just the basic opertations such as \texttt{+}, \texttt{*} and \texttt{<<}. Unfortuantly about
the only problem this solves is unsoundness (as we can restrict the set of operations allowed). Fitting 
this into the metadata is still an issue, and now we've introduced the problem of having to define this 
language, which is pretty much just a mapping down to CIL bytecodes anyway!

Operations on values exasperates the storage issues again. While above we described using the constant table
metadata to store values and paramaters would just point to that table we can no longer do this. If values can be 
changed then we are no longer working with constants, and we can't solve this problem in the same was as C++ templates 
(expanding and running all template operations at compile time) because we have to support dynamic loading and linking.

If our \texttt{Array} type from above was in a library then we need to record in the metadata that the return type
of \texttt{Concatenate} is \texttt{length + n}. We have no way of collapsing that to a single constant as \texttt{Array}
has not yet been instantiated.

\begin{lstlisting}[caption={Using Array},keywordstyle={\color{blue}},language=sharpc]
Array<int, 4> arr1 = new Array<int, 4>(new int[] { 1, 2, 3, 4});
Array<int, 2> arr2 = new Array<int, 2>(new int[] { 5, 6});
Array<int, 6> arr3 = arr1.Concatenate<2>(arr2); 
// in all likelihood the <2> would be infered by high level languges
\end{lstlisting}


\subsection{Variable structures}

As described in our motivating example, one use case of dependent types is for vector types.

We want these vectors to be value types with all their data allocated inline. This makes for fast and local
allocations which is good for cache coherency, but also makes it possible to easily transfer them to native
APIs such as OpenGL and OpenCL.

The standard CLI can define types with a fixed size, either via multiple fields and explicit layout or via explicit size.
Neither of these allow a types inline size to be changed per use case, the only dynamically sized types in the CLI are
arrays which are allocated on the heap.

\begin{lstlisting}[caption={Vector using CLI array}, keywordstyle={\color{blue}},language=sharpc]
public struct Vector
{
	public readonly float[] values;
	
	public Vector(int n) {
		values = new float[n];
	}
	
	...
}
\end{lstlisting}

While this allows vectors of different dimensions, it is not cache efficient, easy to copy to native APIs or immutable.
It also requires many runtime checks to make sure that the vectors passed into methods are the right size.

\begin{lstlisting}[caption={Runtime checks needed when using arrays}, keywordstyle={\color{blue}},language=sharpc]
public static float Dot(Vector a, Vector b)
{
	if(a.values.length != b.values.length)
		throw new ArgumentException("a and b are different sizes.");
		
	float dot = 0;
	for(int i=0; i<a.Length; ++i)
	{
		dot += a.values[i]*b.values[i];
	}
	return dot;
}
\end{lstlisting}

C\# allows the declaration of fixed size arrays. These are implemented via an internal, explicitly sized class and pointer
arithmetic. Due to the pointer arithmetic they are only allowed in an unsafe context (a C\# construct to delimit unsafe code).
The size of the fixed array must be a compile time constant. Given the fixed size and the generally unwanted requirement for
unsafe code, fixed arrays don't have many uses. Certainly for these small vector types (which are often accessed for a specific 
field such as \texttt{X}) it's better to use multiple fields.

\begin{lstlisting}[caption={A fixed size vector type}, keywordstyle={\color{blue}},language=sharpc]
public unsafe struct Vector3
{
	public fixed float values[3]; 
	
	...
}
\end{lstlisting}

\begin{lstlisting}[caption={Translation to CIL}, keywordstyle={\color{blue}},language=cil]
.class sequential ansi sealed nested public beforefieldinit Vector3
    extends [mscorlib]System.ValueType
{
    .field public valuetype Test.Program/Vector3/<Values>e__FixedBuffer0 Values
    {
        .custom instance void [mscorlib]System.Runtime.CompilerServices.FixedBufferAttribute::.ctor(class [mscorlib]System.Type, int32) = 
			{ type(float32) int32(3) }
    }

    .class sequential ansi sealed nested public beforefieldinit <Values>e__FixedBuffer0
        extends [mscorlib]System.ValueType
    {
        .custom instance void [mscorlib]System.Runtime.CompilerServices.UnsafeValueTypeAttribute::.ctor()
        .custom instance void [mscorlib]System.Runtime.CompilerServices.CompilerGeneratedAttribute::.ctor()
        .field public float32 FixedElementField

    }
}
\end{lstlisting}

If we introduce values as type paramters we want some way of adding or removing data based on that value. A \texttt{Vector<2>} 
should only take up 2 floats of storage. 

\begin{lstlisting}[caption={A fixed size vector type}, keywordstyle={\color{blue}},language=sharpc]
public struct Vector<int n>
{
	//fixed buffers are currently supported in C\#,
	//but their size must be a compile time constant
	public fixed float values[n]; 
	
	...
}
\end{lstlisting}

There seems two immediate ways to solve this problem. Firstly by allowing a reference to the value parameter in the 
\texttt{FixedBufferAttribute} or by adding fixed size buffers to the CLI as a supported type.

\chapter{Evaluation}

\section{Type equality constraints}

Our specification of type equality constraints extends the expressibility of the CLI. 
We feel that having this expressible at the CLI level (and not just the language) level
is beneficial, it means it's easier to put to use in languages that target the CLI as 
the work require to make it run correctly has already been done, and the runtime metadata 
available from reflection can more accurately match the expressions at the language level.

There are a number of positive points about the final design of this feature, most importantly 
that it is a fairly small change to the CLI system requiring only a few localized modifications 
to method calls and the specification of assignment compatibility. The fact that the changes to
assignment compatibility cascade to affect the rest of the system in the correct way is a neat 
feature. The modifications required in each location are themselves fairly small additions, and 
notably no subtractions so all old code would continue to behave the same way in the new system.
While it has not been possible to implement the system in Mono we have described which parts of 
Mono will need to be modified to support each part of the new system.

The system also has some room for extension thanks to the flags field in the metadata table. 
While currently this is defined to always be zero, future work could add other type relationships.

The final positive point to make is that we have describe a versatile test system that could be
used by any implementation to help check that the implementation behaves correctly.

This work has not gone perfectly however, a number of ideas we had at the start of the project 
have not been brought to fruition. Most major of these is the lack of any implementation, but 
we also have no description of how to do match-and-bind.

Match-and-bind requires a fairly hefty transform step from a high level language to CIL, we had 
hoped to define an instruction that would do this work for the compiler writer so they merely had 
to output the one instruction. It would of been a nice addition and is something to consider in any
future work.

The lack of an implementation is a big disappointment. At the start of the project we spent a small amount 
of time looking into Mono and decided that it would be possible to extend. However when the time came 
to start modifying Mono the expanse of the system overwhelmed us. As pointed out above we manage to track down
which parts, functions and files definitely need to be changed, but working out exactly how to incorporate the 
system proved much more difficult than we anticipated.

Finally we have no way to know for sure that the system we have describe is correct. We have found no work 
into formally specifying and proving properties about the CLI. Given no ground work (which would be a significant
project in it's own right) which we could extend off we only have the hope that we have understood the CLI and this
extension well enough to not have made any major mistakes.

\section{Value dependent types}

While at the start of this project we had grand plans for value dependent types work on them 
has been difficult. This should of been obvious given how rare implementations of value 
dependence are but we wanted to give it our best shot. 

We still feel that value dependent types would be beneficial to the CLI architecture, despite
the difficulties surrounding them. Our original motivating example (encountered nearly 9 months 
ago) is still relevant. Thus although we have no final design or specification we hope that our discussions
in this report can help move ideas about value dependence forward.

On that note we're pleased with the ideas and issues discussed so far. While it doesn't cover nearly enough 
ground it's starting to expose what needs to be thought about when designing value dependence for the CLI.

One problem that we have continually ran into while thinking about value dependence is the trade off between
an elegant system and a performant system. These often appear to be at conflict with each other resulting in ideas
that we can't consider acceptable due to a shortcoming in one or the other goal.

\chapter{Conclusion}

We've presented a specification for type relationship constraints in the CLI as well as 
as description of how theses could start to be implemented in the Mono runtime. We have 
also discussed ideas on how the CLI could be extended to support values as type parameters
and the issues and advantages these would bring.

\bibliographystyle{plain}
\bibliography{report}

%Still need to fix this. Also need to setup a glossary.%

\end{document}
